% =======================================================
% Έρευνα: Αλγόριθμος vs Πρόγραμμα στη ΓΛΩΣΣΑ
% Βασισμένη στο επίσημο σχολικό βιβλίο:
% "Ανάπτυξη Εφαρμογών σε Προγραμματιστικό Περιβάλλον"
% =======================================================

\documentclass[a4paper,11pt]{article}

% Ελληνικά με XeLaTeX/LuaLaTeX
\usepackage{fontspec}
\usepackage{polyglossia}
\setdefaultlanguage{greek}
\setotherlanguage{english}

% Γραμματοσειρές
\setmainfont{Times New Roman}
\setsansfont{Arial}
\setmonofont{Consolas}

% Πακέτα
\usepackage{geometry}
\geometry{margin=2.5cm}
\usepackage{parskip}
\usepackage{enumitem}
\usepackage{xcolor}
\usepackage{listings}
\usepackage{tcolorbox}
\usepackage{hyperref}

% Χρώματα
\definecolor{codegreen}{rgb}{0,0.6,0}
\definecolor{codegray}{rgb}{0.5,0.5,0.5}
\definecolor{codepurple}{rgb}{0.58,0,0.82}
\definecolor{backcolour}{rgb}{0.95,0.95,0.92}
\definecolor{keywordcolor}{rgb}{0,0,0.8}

% Ρυθμίσεις για κώδικα ΓΛΩΣΣΑΣ
\lstdefinestyle{glossa}{
    backgroundcolor=\color{backcolour},
    commentstyle=\color{codegreen},
    keywordstyle=\color{keywordcolor}\bfseries,
    numberstyle=\tiny\color{codegray},
    stringstyle=\color{codepurple},
    basicstyle=\ttfamily\small,
    breakatwhitespace=false,
    breaklines=true,
    captionpos=b,
    keepspaces=true,
    numbers=left,
    numbersep=5pt,
    showspaces=false,
    showstringspaces=false,
    showtabs=false,
    tabsize=4,
    frame=single,
    rulecolor=\color{black},
}
\lstset{style=glossa}

% Hyperref ρυθμίσεις
\hypersetup{
    colorlinks=true,
    linkcolor=blue,
    filecolor=magenta,
    urlcolor=cyan,
}

% =======================================================
\begin{document}

% Τίτλος
\begin{center}
    {\LARGE\bfseries Αλγόριθμος vs Πρόγραμμα στη ΓΛΩΣΣΑ}\\[0.5cm]
    {\large Συγκριτική Ανάλυση}\\[0.3cm]
    {\normalsize Βασισμένη στο σχολικό βιβλίο ΑΕΠΠ}
\end{center}

\vspace{1cm}

\tableofcontents
\newpage

% =======================================================
\section{Εισαγωγή}
% =======================================================

Στο μάθημα «Ανάπτυξη Εφαρμογών σε Προγραμματιστικό Περιβάλλον» (ΑΕΠΠ), υπάρχουν δύο βασικές δομές που χρησιμοποιούμε:

\begin{itemize}
    \item \textbf{Αλγόριθμος:} Θεωρητική αναπαράσταση της λύσης ενός προβλήματος (ψευδοκώδικας)
    \item \textbf{Πρόγραμμα:} Υλοποίηση του αλγορίθμου στη ΓΛΩΣΣΑ (εκτελέσιμος κώδικας)
\end{itemize}

\begin{tcolorbox}[colback=blue!5,colframe=blue!50!black,title=Σημαντική Διάκριση]
Ο \textbf{Αλγόριθμος} είναι \underline{θεωρητική έννοια} που περιγράφει βήματα επίλυσης προβλήματος.\\
Το \textbf{Πρόγραμμα} είναι η \underline{υλοποίηση} του αλγορίθμου σε συγκεκριμένη γλώσσα προγραμματισμού.
\end{tcolorbox}

% =======================================================
\section{Η Δομή του Αλγορίθμου (Ψευδοκώδικας)}
% =======================================================

\subsection{Βασική Μορφή}

Ο αλγόριθμος στο σχολικό βιβλίο γράφεται ως εξής:

\begin{lstlisting}[language=,morekeywords={Αλγόριθμος,Τέλος},title=Γενική δομή αλγορίθμου]
Αλγόριθμος ΌνομαΑλγορίθμου
  εντολές
  ...
Τέλος ΌνομαΑλγορίθμου
\end{lstlisting}

\subsection{Παράδειγμα Αλγορίθμου}

\begin{lstlisting}[language=,morekeywords={Αλγόριθμος,Τέλος,Διάβασε,Γράψε,Αν,τότε,αλλιώς,Τέλος_αν},title=Παράδειγμα: Εύρεση μεγαλύτερου αριθμού]
Αλγόριθμος ΜεγαλύτεροςΔύο
  Διάβασε α, β
  Αν α > β τότε
    max <- α
  αλλιώς
    max <- β
  Τέλος_αν
  Γράψε max
Τέλος ΜεγαλύτεροςΔύο
\end{lstlisting}

\subsection{Χαρακτηριστικά του Αλγορίθμου}

\begin{center}
\begin{tabular}{|l|l|}
\hline
\textbf{Στοιχείο} & \textbf{Περιγραφή} \\
\hline
Επικεφαλίδα & \texttt{Αλγόριθμος ΌνομαΑλγορίθμου} \\
\hline
Δήλωση μεταβλητών & \textbf{ΔΕΝ απαιτείται} \\
\hline
Δήλωση σταθερών & \textbf{ΔΕΝ απαιτείται} \\
\hline
Τερματισμός & \texttt{Τέλος ΌνομαΑλγορίθμου} \\
\hline
Εκτέλεση & \textbf{ΔΕΝ εκτελείται} σε υπολογιστή \\
\hline
Υποπρογράμματα & \textbf{ΔΕΝ υποστηρίζονται} τυπικά \\
\hline
\end{tabular}
\end{center}

\begin{tcolorbox}[colback=yellow!10,colframe=orange!50!black,title=Σημαντικό]
Ο αλγόριθμος είναι \textbf{ανεξάρτητος} από γλώσσα προγραμματισμού. Οι εντολές γράφονται σε ελεύθερη μορφή (π.χ. «Διάβασε» με μικρά γράμματα). Δεν απαιτείται αυστηρή σύνταξη.
\end{tcolorbox}

% =======================================================
\section{Η Δομή του Προγράμματος (ΓΛΩΣΣΑ)}
% =======================================================

\subsection{Βασική Μορφή}

Το πρόγραμμα στη ΓΛΩΣΣΑ έχει αυστηρή δομή:

\begin{lstlisting}[language=,morekeywords={ΠΡΟΓΡΑΜΜΑ,ΣΤΑΘΕΡΕΣ,ΜΕΤΑΒΛΗΤΕΣ,ΑΚΕΡΑΙΕΣ,ΠΡΑΓΜΑΤΙΚΕΣ,ΧΑΡΑΚΤΗΡΕΣ,ΛΟΓΙΚΕΣ,ΑΡΧΗ,ΤΕΛΟΣ_ΠΡΟΓΡΑΜΜΑΤΟΣ},title=Γενική δομή προγράμματος]
ΠΡΟΓΡΑΜΜΑ ΌνομαΠρογράμματος
ΣΤΑΘΕΡΕΣ
  ΌνομαΣταθ = τιμή
ΜΕΤΑΒΛΗΤΕΣ
  ΤΥΠΟΣ: όνομα1, όνομα2, ...
ΑΡΧΗ
  εντολές
  ...
ΤΕΛΟΣ_ΠΡΟΓΡΑΜΜΑΤΟΣ
\end{lstlisting}

\subsection{Παράδειγμα Προγράμματος}

\begin{lstlisting}[language=,morekeywords={ΠΡΟΓΡΑΜΜΑ,ΜΕΤΑΒΛΗΤΕΣ,ΑΚΕΡΑΙΕΣ,ΑΡΧΗ,ΔΙΑΒΑΣΕ,ΓΡΑΨΕ,ΑΝ,ΤΟΤΕ,ΑΛΛΙΩΣ,ΤΕΛΟΣ_ΑΝ,ΤΕΛΟΣ_ΠΡΟΓΡΑΜΜΑΤΟΣ},title=Παράδειγμα: Εύρεση μεγαλύτερου αριθμού]
ΠΡΟΓΡΑΜΜΑ ΜεγαλύτεροςΔύο
ΜΕΤΑΒΛΗΤΕΣ
  ΑΚΕΡΑΙΕΣ: α, β, max
ΑΡΧΗ
  ΔΙΑΒΑΣΕ α, β
  ΑΝ α > β ΤΟΤΕ
    max <- α
  ΑΛΛΙΩΣ
    max <- β
  ΤΕΛΟΣ_ΑΝ
  ΓΡΑΨΕ max
ΤΕΛΟΣ_ΠΡΟΓΡΑΜΜΑΤΟΣ
\end{lstlisting}

\subsection{Χαρακτηριστικά του Προγράμματος}

\begin{center}
\begin{tabular}{|l|l|}
\hline
\textbf{Στοιχείο} & \textbf{Περιγραφή} \\
\hline
Επικεφαλίδα & \texttt{ΠΡΟΓΡΑΜΜΑ ΌνομαΠρογράμματος} \\
\hline
Δήλωση μεταβλητών & \textbf{ΥΠΟΧΡΕΩΤΙΚΗ} (αν υπάρχουν) \\
\hline
Δήλωση σταθερών & Προαιρετική (αν υπάρχουν) \\
\hline
Κυρίως σώμα & \texttt{ΑΡΧΗ ... ΤΕΛΟΣ\_ΠΡΟΓΡΑΜΜΑΤΟΣ} \\
\hline
Εκτέλεση & \textbf{ΕΚΤΕΛΕΙΤΑΙ} στο Διερμηνευτή \\
\hline
Υποπρογράμματα & \textbf{Υποστηρίζονται} (Διαδικασίες, Συναρτήσεις) \\
\hline
\end{tabular}
\end{center}

% =======================================================
\section{Συγκριτικός Πίνακας Διαφορών}
% =======================================================

\begin{center}
\begin{tabular}{|l|c|c|}
\hline
\textbf{Χαρακτηριστικό} & \textbf{Αλγόριθμος} & \textbf{Πρόγραμμα} \\
\hline
\hline
\multicolumn{3}{|c|}{\textbf{ΔΟΜΗ}} \\
\hline
Επικεφαλίδα & \texttt{Αλγόριθμος} & \texttt{ΠΡΟΓΡΑΜΜΑ} \\
\hline
Τερματισμός & \texttt{Τέλος} & \texttt{ΤΕΛΟΣ\_ΠΡΟΓΡΑΜΜΑΤΟΣ} \\
\hline
Κυρίως σώμα & Άμεσα μετά την επικεφαλίδα & Μέσα σε \texttt{ΑΡΧΗ...ΤΕΛΟΣ\_ΠΡΟΓΡΑΜΜΑΤΟΣ} \\
\hline
\hline
\multicolumn{3}{|c|}{\textbf{ΔΗΛΩΣΕΙΣ}} \\
\hline
Δήλωση μεταβλητών & \textbf{ΟΧΙ} & \textbf{ΝΑΙ} (υποχρεωτική) \\
\hline
Δήλωση σταθερών & \textbf{ΟΧΙ} & \textbf{ΝΑΙ} (προαιρετική) \\
\hline
Τύποι δεδομένων & Υπονοούνται & Δηλώνονται ρητά \\
\hline
\hline
\multicolumn{3}{|c|}{\textbf{ΣΥΝΤΑΞΗ}} \\
\hline
Αυστηρή σύνταξη & \textbf{ΟΧΙ} (ελεύθερη) & \textbf{ΝΑΙ} (αυστηρή) \\
\hline
Κεφαλαία/Πεζά & Ελεύθερα & Λέξεις-κλειδιά ΚΕΦΑΛΑΙΑ \\
\hline
Ανάθεση τιμής & \texttt{<-} ή $\leftarrow$ & \texttt{<-} \\
\hline
\hline
\multicolumn{3}{|c|}{\textbf{ΕΚΤΕΛΕΣΗ}} \\
\hline
Εκτελέσιμο & \textbf{ΟΧΙ} & \textbf{ΝΑΙ} \\
\hline
Διερμηνευτής & Δεν τρέχει & Τρέχει \\
\hline
Αποσφαλμάτωση & Χειροκίνητη & Με εργαλεία \\
\hline
\hline
\multicolumn{3}{|c|}{\textbf{ΥΠΟΠΡΟΓΡΑΜΜΑΤΑ}} \\
\hline
Διαδικασίες & \textbf{ΟΧΙ} τυπικά & \textbf{ΝΑΙ} \\
\hline
Συναρτήσεις & \textbf{ΟΧΙ} τυπικά & \textbf{ΝΑΙ} \\
\hline
Εμφωλευμένοι αλγόριθμοι & Σε θεωρία & Με ΚΑΛΕΣΕ \\
\hline
\end{tabular}
\end{center}

% =======================================================
\section{Τι μπορεί να κάνει το ένα που δεν μπορεί το άλλο}
% =======================================================

\subsection{Τι μπορεί ΜΟΝΟ ο Αλγόριθμος}

\begin{enumerate}
    \item \textbf{Ελεύθερη περιγραφή:} Μπορεί να γραφεί με φυσική γλώσσα ή διαγράμματα ροής
    \item \textbf{Αφηρημένη σκέψη:} Δεν χρειάζεται να προσδιοριστούν τύποι δεδομένων
    \item \textbf{Γενικότητα:} Περιγράφει τη λογική χωρίς τους περιορισμούς συγκεκριμένης γλώσσας
    \item \textbf{Θεωρητική ανάλυση:} Μπορεί να αναλυθεί η πολυπλοκότητα χωρίς υλοποίηση
\end{enumerate}

\subsection{Τι μπορεί ΜΟΝΟ το Πρόγραμμα}

\begin{enumerate}
    \item \textbf{Εκτέλεση:} Μπορεί να εκτελεστεί και να δώσει πραγματικά αποτελέσματα
    \item \textbf{Έλεγχος σφαλμάτων:} Ο Διερμηνευτής εντοπίζει συντακτικά/λογικά λάθη
    \item \textbf{Υποπρογράμματα:}
    \begin{itemize}
        \item \texttt{ΔΙΑΔΙΚΑΣΙΑ ... ΤΕΛΟΣ\_ΔΙΑΔΙΚΑΣΙΑΣ}
        \item \texttt{ΣΥΝΑΡΤΗΣΗ ... ΤΕΛΟΣ\_ΣΥΝΑΡΤΗΣΗΣ}
    \end{itemize}
    \item \textbf{Χρήση πινάκων:} Με αυστηρή δήλωση διαστάσεων
    \item \textbf{Ενσωματωμένες συναρτήσεις:} \texttt{Α\_Μ}, \texttt{Α\_Τ}, \texttt{Τ\_Ρ}, κ.λπ.
    \item \textbf{Αρχεία:} Δυνατότητα εισόδου/εξόδου με αρχεία
\end{enumerate}

% =======================================================
\section{Ο Ρόλος των Δηλώσεων}
% =======================================================

\subsection{Αλγόριθμος: Χωρίς Δηλώσεις}

Στον αλγόριθμο \textbf{δεν δηλώνουμε μεταβλητές}. Τις χρησιμοποιούμε απευθείας:

\begin{lstlisting}[language=,morekeywords={Αλγόριθμος,Τέλος,Διάβασε,Γράψε},title=Αλγόριθμος χωρίς δηλώσεις]
Αλγόριθμος ΆθροισμαΔύο
  Διάβασε α, β
  άθροισμα <- α + β
  Γράψε άθροισμα
Τέλος ΆθροισμαΔύο
\end{lstlisting}

Παρατηρήστε ότι οι μεταβλητές \texttt{α}, \texttt{β}, \texttt{άθροισμα} χρησιμοποιούνται χωρίς προηγούμενη δήλωση του τύπου τους.

\subsection{Πρόγραμμα: Υποχρεωτικές Δηλώσεις}

Στο πρόγραμμα \textbf{ΠΡΕΠΕΙ} να δηλώσουμε κάθε μεταβλητή με τον τύπο της:

\begin{lstlisting}[language=,morekeywords={ΠΡΟΓΡΑΜΜΑ,ΜΕΤΑΒΛΗΤΕΣ,ΑΚΕΡΑΙΕΣ,ΑΡΧΗ,ΔΙΑΒΑΣΕ,ΓΡΑΨΕ,ΤΕΛΟΣ_ΠΡΟΓΡΑΜΜΑΤΟΣ},title=Πρόγραμμα με υποχρεωτικές δηλώσεις]
ΠΡΟΓΡΑΜΜΑ ΆθροισμαΔύο
ΜΕΤΑΒΛΗΤΕΣ
  ΑΚΕΡΑΙΕΣ: α, β, άθροισμα
ΑΡΧΗ
  ΔΙΑΒΑΣΕ α, β
  άθροισμα <- α + β
  ΓΡΑΨΕ άθροισμα
ΤΕΛΟΣ_ΠΡΟΓΡΑΜΜΑΤΟΣ
\end{lstlisting}

\begin{tcolorbox}[colback=red!10,colframe=red!50!black,title=Προσοχή]
Αν χρησιμοποιήσουμε μεταβλητή στο πρόγραμμα χωρίς να την έχουμε δηλώσει, ο Διερμηνευτής θα εμφανίσει \textbf{σφάλμα μεταγλώττισης}.
\end{tcolorbox}

\subsection{Τύποι Δηλώσεων στο Πρόγραμμα}

\begin{center}
\begin{tabular}{|l|l|l|}
\hline
\textbf{Δήλωση} & \textbf{Λέξη-κλειδί} & \textbf{Παράδειγμα} \\
\hline
Ακέραιοι & \texttt{ΑΚΕΡΑΙΕΣ} & \texttt{ΑΚΕΡΑΙΕΣ: α, β, γ} \\
\hline
Πραγματικοί & \texttt{ΠΡΑΓΜΑΤΙΚΕΣ} & \texttt{ΠΡΑΓΜΑΤΙΚΕΣ: x, y} \\
\hline
Χαρακτήρες & \texttt{ΧΑΡΑΚΤΗΡΕΣ} & \texttt{ΧΑΡΑΚΤΗΡΕΣ: όνομα, πόλη} \\
\hline
Λογικές & \texttt{ΛΟΓΙΚΕΣ} & \texttt{ΛΟΓΙΚΕΣ: βρέθηκε, τέλος} \\
\hline
Πίνακες & \texttt{ΤΥΠΟΣ: Α[Ν]} & \texttt{ΑΚΕΡΑΙΕΣ: Πίνακας[100]} \\
\hline
\end{tabular}
\end{center}

% =======================================================
\section{Περιορισμοί της ΓΛΩΣΣΑΣ}
% =======================================================

\subsection{Περιορισμός στη ΔΙΑΒΑΣΕ}

\begin{tcolorbox}[colback=red!10,colframe=red!50!black,title=Σημαντικός Περιορισμός]
Η εντολή \texttt{ΔΙΑΒΑΣΕ} \textbf{ΔΕΝ} μπορεί να διαβάσει \textbf{ΛΟΓΙΚΕΣ} μεταβλητές!
\end{tcolorbox}

Για να «διαβάσουμε» μια λογική τιμή, χρησιμοποιούμε έναν ακέραιο και μετατροπή:

\begin{lstlisting}[language=,morekeywords={ΠΡΟΓΡΑΜΜΑ,ΜΕΤΑΒΛΗΤΕΣ,ΑΚΕΡΑΙΕΣ,ΛΟΓΙΚΕΣ,ΑΡΧΗ,ΔΙΑΒΑΣΕ,ΑΝ,ΤΟΤΕ,ΑΛΛΙΩΣ,ΤΕΛΟΣ_ΑΝ,ΑΛΗΘΗΣ,ΨΕΥΔΗΣ,ΤΕΛΟΣ_ΠΡΟΓΡΑΜΜΑΤΟΣ},title=Τρόπος ανάγνωσης λογικής τιμής]
ΠΡΟΓΡΑΜΜΑ ΔιάβασμαΛογικής
ΜΕΤΑΒΛΗΤΕΣ
  ΑΚΕΡΑΙΕΣ: απάντηση
  ΛΟΓΙΚΕΣ: επιλογή
ΑΡΧΗ
  ΔΙΑΒΑΣΕ απάντηση     ! Διαβάζουμε ακέραιο (1=ναι, 0=όχι)
  ΑΝ απάντηση = 1 ΤΟΤΕ
    επιλογή <- ΑΛΗΘΗΣ
  ΑΛΛΙΩΣ
    επιλογή <- ΨΕΥΔΗΣ
  ΤΕΛΟΣ_ΑΝ
ΤΕΛΟΣ_ΠΡΟΓΡΑΜΜΑΤΟΣ
\end{lstlisting}

\subsection{Άλλοι Περιορισμοί}

\begin{itemize}
    \item Οι συναρτήσεις \textbf{δεν} μπορούν να επιστρέψουν πίνακες
    \item Οι παράμετροι συναρτήσεων περνιούνται με \textbf{τιμή} (όχι με αναφορά)
    \item Οι παράμετροι διαδικασιών περνιούνται με \textbf{αναφορά}
    \item \texttt{DIV} και \texttt{MOD} δουλεύουν \textbf{μόνο} με ακέραιους
\end{itemize}

% =======================================================
\section{Συμπεράσματα}
% =======================================================

\begin{tcolorbox}[colback=green!10,colframe=green!50!black,title=Βασικά Συμπεράσματα]
\begin{enumerate}
    \item Ο \textbf{Αλγόριθμος} είναι το \textbf{«σχέδιο»} - περιγράφει τη λογική επίλυσης
    \item Το \textbf{Πρόγραμμα} είναι η \textbf{«κατασκευή»} - υλοποιεί τον αλγόριθμο
    \item Στον αλγόριθμο \textbf{ΔΕΝ} δηλώνουμε μεταβλητές
    \item Στο πρόγραμμα \textbf{ΠΡΕΠΕΙ} να δηλώσουμε τύπους δεδομένων
    \item Το πρόγραμμα \textbf{υποστηρίζει} υποπρογράμματα (διαδικασίες, συναρτήσεις)
    \item Μόνο το πρόγραμμα \textbf{εκτελείται} στον υπολογιστή
\end{enumerate}
\end{tcolorbox}

\vspace{1cm}

\begin{center}
\rule{0.5\textwidth}{0.4pt}\\[0.5cm]
\textit{Έγγραφο δημιουργημένο για εκπαιδευτικούς σκοπούς}\\
\textit{ΑΕΠΠ - Γ' Λυκείου}
\end{center}

\end{document}
