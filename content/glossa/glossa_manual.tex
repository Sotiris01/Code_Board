% =======================================================
% Το Εγχειρίδιο της ΓΛΩΣΣΑΣ
% Ανάπτυξη Εφαρμογών σε Προγραμματιστικό Περιβάλλον
% =======================================================

\documentclass[a4paper,11pt]{article}

% Ελληνικά με XeLaTeX/LuaLaTeX
\usepackage{fontspec}
\usepackage{polyglossia}
\setdefaultlanguage{greek}
\setotherlanguage{english}

% Γραμματοσειρές
\setmainfont{Times New Roman}
\setsansfont{Arial}
\setmonofont{Consolas}

% Πακέτα
\usepackage{geometry}
\geometry{margin=2.5cm}
\usepackage{parskip}
\usepackage{enumitem}
\usepackage{xcolor}
\usepackage{listings}
\usepackage{tcolorbox}
\usepackage{hyperref}

% Χρώματα
\definecolor{codegreen}{rgb}{0,0.6,0}
\definecolor{codegray}{rgb}{0.5,0.5,0.5}
\definecolor{codepurple}{rgb}{0.58,0,0.82}
\definecolor{backcolour}{rgb}{0.95,0.95,0.92}
\definecolor{keywordcolor}{rgb}{0,0,0.8}

% Ρυθμίσεις για κώδικα ΓΛΩΣΣΑΣ
\lstdefinestyle{glossa}{
    backgroundcolor=\color{backcolour},
    commentstyle=\color{codegreen},
    keywordstyle=\color{keywordcolor}\bfseries,
    numberstyle=\tiny\color{codegray},
    stringstyle=\color{codepurple},
    basicstyle=\ttfamily\small,
    breakatwhitespace=false,
    breaklines=true,
    captionpos=b,
    keepspaces=true,
    numbers=left,
    numbersep=5pt,
    showspaces=false,
    showstringspaces=false,
    showtabs=false,
    tabsize=4,
    frame=single,
    rulecolor=\color{black},
}
\lstset{style=glossa}

% Hyperref ρυθμίσεις
\hypersetup{
    colorlinks=true,
    linkcolor=blue,
    filecolor=magenta,
    urlcolor=cyan,
}

% =======================================================
\begin{document}

% Τίτλος
\begin{center}
    {\LARGE\bfseries Το Εγχειρίδιο της ΓΛΩΣΣΑΣ}\\[0.5cm]
    {\large Ανάπτυξη Εφαρμογών σε Προγραμματιστικό Περιβάλλον}\\[0.3cm]
    {\normalsize Γ' Λυκείου}
\end{center}

\vspace{1cm}

\tableofcontents
\newpage

% =======================================================
\section{Οι τύποι δεδομένων της ΓΛΩΣΣΑΣ}
% =======================================================

Όταν λέμε \textbf{τύπο δεδομένων}, εννοούμε από ποιο σύνολο τιμών παίρνει τιμές μια έκφραση. Στη ΓΛΩΣΣΑ υπάρχουν τέσσερις τύποι δεδομένων:

\begin{itemize}
    \item ο \textbf{ακέραιος}, π.χ. \texttt{-1234567890}
    \item ο \textbf{πραγματικός}, π.χ. \texttt{3.14}
    \item οι \textbf{χαρακτήρες}, π.χ. \texttt{"Ν"} ή \texttt{'Μυρτώ'}
    \item ο \textbf{λογικός}, δηλαδή \texttt{ΑΛΗΘΗΣ} ή \texttt{ΨΕΥΔΗΣ}
\end{itemize}

\subsection{Παρατηρήσεις}

\begin{enumerate}[label=\arabic*.]
    \item \textbf{Όρια ακεραίων και πραγματικών:} Αν και προγραμματίζοντας στο χαρτί δεν αντιμετωπίζουμε πρόβλημα μεγέθους στους αριθμούς, στον υπολογιστή έχουμε πάντα περιορισμένο μέγεθος, όσο μεγάλο και αν είναι αυτό. Στην τρέχουσα υλοποίηση του Διερμηνευτή:
    \begin{itemize}
        \item Οι \textbf{ακέραιοι} είναι από $-9.223.372.036.854.775.808$ μέχρι $9.223.372.036.854.775.807$ (64 bit προσημασμένος)
        \item Οι \textbf{πραγματικοί} από $5.0 \times 10^{-324}$ μέχρι $1.7 \times 10^{308}$
    \end{itemize}

    \item \textbf{Ακρίβεια πραγματικών:} Πρόβλημα μεγέθους υπάρχει και στους πολύ μικρούς πραγματικούς αριθμούς. Δηλαδή ενώ στο σύνολο των πραγματικών αριθμών δε νοείται ο επόμενος του 1, στον υπολογιστή υπάρχει πάντα ο επόμενος του 1, και στην τρέχουσα υλοποίηση του Διερμηνευτή είναι ο $5.0 \times 10^{-324}$. Έτσι αν για παράδειγμα προσπαθήσουμε να προσθέσουμε έναν πολύ μικρό αριθμό στο 1, το άθροισμα θα είναι πάλι 1! Όταν το αποτέλεσμα κάποιας πράξης είναι τόσο μικρός αριθμός που δεν μπορεί να αναπαρασταθεί από τον υπολογιστή, προκύπτει \textbf{λάθος εκτέλεσης}.

    \item \textbf{Χαρακτήρες και αλφαριθμητικά:} Δεν υπάρχει διαφορετικός τύπος δεδομένων για ένα χαρακτήρα και για πολλούς χαρακτήρες (π.χ. \texttt{char} και \texttt{string}). Η σύγκριση μεταξύ των αλφαριθμητικών γίνεται σύμφωνα με τη διάταξη του προτύπου Unicode, και έτσι:
    \begin{itemize}
        \item Τα κεφαλαία είναι διαφορετικά από τα πεζά
        \item Τα τονισμένα διαφορετικά από τα άτονα
    \end{itemize}

    \item \textbf{Εισαγωγικά σε αλφαριθμητικά:} Τα αλφαριθμητικά επιτρέπεται να περικλείονται είτε με απλά είτε με διπλά εισαγωγικά. Για να συμπεριλάβουμε ένα απλό εισαγωγικό σε ένα αλφαριθμητικό πρέπει να γράψουμε δύο απλά, π.χ.:
    
    \begin{tcolorbox}[colback=backcolour,colframe=black,title=Παράδειγμα εισαγωγικών]
    \texttt{'Γιάννενα πρώτα στ'' άρματα'}\\
    ή\\
    \texttt{"Γιάννενα πρώτα στ' άρματα"}
    \end{tcolorbox}
    
    Το αντίστροφο συμβαίνει για τα αλφαριθμητικά που περικλείονται από διπλά εισαγωγικά.

    \item \textbf{Προβλήματα ακρίβειας σε επαναλήψεις:} Οι πραγματικοί αριθμοί λόγω της περιορισμένης ακρίβειας δε συμπεριφέρονται πάντα όπως θα θέλαμε. Για παράδειγμα αν σε οποιαδήποτε γλώσσα προγραμματισμού προσθέσουμε στο 1 τριάντα φορές το 0.30 δεν θα πάρουμε το 10 αλλά το 10.000000001.
    
    Αυτό δημιουργεί προβλήματα σε εντολές επανάληψης, αφού μπορεί να γίνει μία επανάληψη λιγότερη ή περισσότερη. Ο Διερμηνευτής αντιμετωπίζει αυτό το πρόβλημα «περιορίζοντας» την ακρίβεια των πραγματικών σε συγκρίσεις.
    
    \begin{tcolorbox}[colback=yellow!10,colframe=orange!50!black,title=Σημείωση]
    Για περισσότερες λεπτομέρειες δείτε την περιγραφή της επιλογής «Σημαντικά δεκαδικά ψηφία πραγματικών σε συγκρίσεις» στο αρχείο βοήθειας του Διερμηνευτή.
    \end{tcolorbox}
\end{enumerate}

\subsection{Συνοπτικός Πίνακας Τύπων Δεδομένων}

\begin{center}
\begin{tabular}{|l|l|l|}
\hline
\textbf{Τύπος} & \textbf{Δήλωση} & \textbf{Παράδειγμα} \\
\hline
Ακέραιος & \texttt{ΑΚΕΡΑΙΕΣ:} & \texttt{-1234567890}, \texttt{42} \\
\hline
Πραγματικός & \texttt{ΠΡΑΓΜΑΤΙΚΕΣ:} & \texttt{3.14}, \texttt{-0.5} \\
\hline
Χαρακτήρες & \texttt{ΧΑΡΑΚΤΗΡΕΣ:} & \texttt{'Μυρτώ'}, \texttt{"Ν"} \\
\hline
Λογικός & \texttt{ΛΟΓΙΚΕΣ:} & \texttt{ΑΛΗΘΗΣ}, \texttt{ΨΕΥΔΗΣ} \\
\hline
\end{tabular}
\end{center}

% =======================================================
\section{Οι τελεστές της ΓΛΩΣΣΑΣ}
% =======================================================

\subsection{Αριθμητικοί τελεστές}

\begin{itemize}
    \item \textbf{\texttt{+} (συν):} Ισχύει για τους ακέραιους και τους πραγματικούς.
    
    \begin{tcolorbox}[colback=yellow!10,colframe=orange!50!black,title=Σημείωση]
    Προαιρετικά και χωρίς να προτείνεται, μπορείτε να ενεργοποιήσετε τη σχετική επιλογή του Διερμηνευτή ώστε να επιτραπεί η συνένωση χαρακτήρων με τον τελεστή \texttt{+}.
    \end{tcolorbox}
    
    \item \textbf{\texttt{-} (πλην):} Ισχύει για τους ακέραιους και τους πραγματικούς. Υφίσταται σαν δυαδικός (π.χ. \texttt{1 - 2}) και σαν μοναδιαίος (π.χ. \texttt{-1}).
    
    \item \textbf{\texttt{*} (επί):} Ισχύει για τους ακέραιους και τους πραγματικούς.
    
    \item \textbf{\texttt{/} (διά):} Ισχύει για τους ακέραιους και τους πραγματικούς. Το αποτέλεσμα της διαίρεσης είναι \textbf{πάντα πραγματικός}, ακόμα και αν δεν υπάρχουν δεκαδικά ψηφία (π.χ. \texttt{4/2}).
    
    \item \textbf{\texttt{\^{}} (δύναμη):} Ισχύει για τους ακέραιους και τους πραγματικούς. Το αποτέλεσμα είναι συνήθως πραγματικός αριθμός, εκτός αν ο εκθέτης είναι θετικός ακέραιος σταθερής αποτίμησης.
    
    \begin{tcolorbox}[colback=backcolour,colframe=black,title=Παραδείγματα δύναμης]
    \texttt{2\^{}2} $\rightarrow$ ακέραιος (= 4)\\
    \texttt{2\^{}-2} $\rightarrow$ πραγματικός (= 0.25)\\
    \texttt{2\^{}ι} $\rightarrow$ πραγματικός (γενική περίπτωση)
    \end{tcolorbox}
    
    \item \textbf{\texttt{DIV} (πηλίκο ακέραιας διαίρεσης):} Ισχύει \textbf{μόνο} για τους ακέραιους.\\
    Παράδειγμα: \texttt{7 DIV 2 = 3} (και περισσεύει 1)
    
    \item \textbf{\texttt{MOD} (υπόλοιπο ακέραιας διαίρεσης):} Ισχύει \textbf{μόνο} για τους ακέραιους.\\
    Παράδειγμα: \texttt{7 MOD 2 = 1}
\end{itemize}

\subsubsection{Συνοπτικός Πίνακας Αριθμητικών Τελεστών}

\begin{center}
\begin{tabular}{|c|l|l|l|}
\hline
\textbf{Τελεστής} & \textbf{Λειτουργία} & \textbf{Τύποι} & \textbf{Αποτέλεσμα} \\
\hline
\texttt{+} & Πρόσθεση & Ακέραιοι, Πραγματικοί & Ανάλογα \\
\hline
\texttt{-} & Αφαίρεση & Ακέραιοι, Πραγματικοί & Ανάλογα \\
\hline
\texttt{*} & Πολλαπλασιασμός & Ακέραιοι, Πραγματικοί & Ανάλογα \\
\hline
\texttt{/} & Διαίρεση & Ακέραιοι, Πραγματικοί & Πραγματικός \\
\hline
\texttt{\^{}} & Δύναμη & Ακέραιοι, Πραγματικοί & Συνήθως Πραγματικός \\
\hline
\texttt{DIV} & Ακέραια διαίρεση & Μόνο Ακέραιοι & Ακέραιος \\
\hline
\texttt{MOD} & Υπόλοιπο & Μόνο Ακέραιοι & Ακέραιος \\
\hline
\end{tabular}
\end{center}

\subsection{Συγκριτικοί τελεστές}

Γενικά οι συγκριτικοί τελεστές εφαρμόζονται μεταξύ των \textbf{ίδιων τύπων δεδομένων} (π.χ. χαρακτήρες $>=$ χαρακτήρες) με εξαίρεση τους ακέραιους και τους πραγματικούς, που μπορεί να συγκρίνονται και μεταξύ τους (π.χ. ακέραιος $=$ πραγματικός). Το αποτέλεσμα των συγκριτικών τελεστών είναι \textbf{πάντα λογικού τύπου}.

\begin{itemize}
    \item \textbf{\texttt{<=} ή \texttt{≤} (μικρότερο ή ίσο):} Ισχύει για τους ακέραιους, τους πραγματικούς και τους χαρακτήρες.
    
    \begin{tcolorbox}[colback=backcolour,colframe=black,title=Διάταξη χαρακτήρων (Unicode)]
    \texttt{A} (αγγλικό) $<$ \texttt{B} $<$ \texttt{a} $<$ \texttt{b} $<$ \texttt{Ά} $<$ \texttt{Α} (ελληνικό) $<$ \texttt{Β} $<$ \texttt{ά} $<$ \texttt{α} $<$ \texttt{β}
    \end{tcolorbox}
    
    \item \textbf{\texttt{<} (μικρότερο):} Ισχύει για τους ακέραιους, τους πραγματικούς και τους χαρακτήρες.
    
    \begin{tcolorbox}[colback=yellow!10,colframe=orange!50!black,title=Σημείωση για λογικές τιμές]
    Στις περισσότερες γλώσσες προγραμματισμού οι λογικές τιμές (\texttt{ΑΛΗΘΗΣ} και \texttt{ΨΕΥΔΗΣ}) έχουν διάταξη (ordinal). Το βιβλίο μαθητή αναφέρει ότι \textbf{δεν μπορεί} να εφαρμοστεί ανισότητα μεταξύ λογικών εκφράσεων.
    \end{tcolorbox}
    
    \item \textbf{\texttt{=} (ίσον):} Ισχύει για \textbf{όλους} τους τύπους δεδομένων.
    
    \item \textbf{\texttt{<>} ή \texttt{≠} (διάφορο):} Ισχύει για \textbf{όλους} τους τύπους δεδομένων.
    
    \item \textbf{\texttt{>} (μεγαλύτερο):} Ισχύει για τους ακέραιους, τους πραγματικούς και τους χαρακτήρες.
    
    \item \textbf{\texttt{>=} ή \texttt{≥} (μεγαλύτερο ή ίσο):} Ισχύει για τους ακέραιους, τους πραγματικούς και τους χαρακτήρες.
\end{itemize}

\subsubsection{Συνοπτικός Πίνακας Συγκριτικών Τελεστών}

\begin{center}
\begin{tabular}{|c|c|l|c|c|c|c|}
\hline
\textbf{Τελεστής} & \textbf{Unicode} & \textbf{Λειτουργία} & \textbf{Ακ.} & \textbf{Πραγ.} & \textbf{Χαρ.} & \textbf{Λογ.} \\
\hline
\texttt{<} & & Μικρότερο & \checkmark & \checkmark & \checkmark & \\
\hline
\texttt{<=} & $\leq$ & Μικρότερο ή ίσο & \checkmark & \checkmark & \checkmark & \\
\hline
\texttt{=} & & Ίσον & \checkmark & \checkmark & \checkmark & \checkmark \\
\hline
\texttt{<>} & $\neq$ & Διάφορο & \checkmark & \checkmark & \checkmark & \checkmark \\
\hline
\texttt{>} & & Μεγαλύτερο & \checkmark & \checkmark & \checkmark & \\
\hline
\texttt{>=} & $\geq$ & Μεγαλύτερο ή ίσο & \checkmark & \checkmark & \checkmark & \\
\hline
\end{tabular}
\end{center}

\subsection{Λογικοί τελεστές}

Οι λογικοί τελεστές (\texttt{ΚΑΙ}, \texttt{Η}, \texttt{ΟΧΙ}) εφαρμόζονται πάνω σε \textbf{λογικές εκφράσεις} (δηλαδή εκφράσεις που το αποτέλεσμά τους είναι \texttt{ΑΛΗΘΗΣ} ή \texttt{ΨΕΥΔΗΣ}). Το αποτέλεσμά τους είναι πάλι λογικού τύπου δεδομένων.

\subsubsection{Πίνακας Αληθείας}

Στον παρακάτω πίνακα θεωρούμε ότι τα \texttt{Α} και \texttt{Β} είναι λογικές εκφράσεις (π.χ. \texttt{1 < 0}, \texttt{β = π}, \texttt{ΑΛΗΘΗΣ}).

\begin{center}
\begin{tabular}{|c|c|c|c|c|}
\hline
\textbf{Α} & \textbf{Β} & \textbf{Α ΚΑΙ Β} & \textbf{Α Η Β} & \textbf{ΟΧΙ Α} \\
\hline
ΨΕΥΔΗΣ & ΨΕΥΔΗΣ & ΨΕΥΔΗΣ & ΨΕΥΔΗΣ & ΑΛΗΘΗΣ \\
\hline
ΨΕΥΔΗΣ & ΑΛΗΘΗΣ & ΨΕΥΔΗΣ & ΑΛΗΘΗΣ & \\
\hline
ΑΛΗΘΗΣ & ΨΕΥΔΗΣ & ΨΕΥΔΗΣ & ΑΛΗΘΗΣ & ΨΕΥΔΗΣ \\
\hline
ΑΛΗΘΗΣ & ΑΛΗΘΗΣ & ΑΛΗΘΗΣ & ΑΛΗΘΗΣ & \\
\hline
\end{tabular}
\end{center}

\begin{itemize}
    \item \textbf{\texttt{ΚΑΙ} (AND):} Επιστρέφει \texttt{ΑΛΗΘΗΣ} μόνο αν \textbf{και οι δύο} τελεστέοι είναι \texttt{ΑΛΗΘΗΣ}.
    \item \textbf{\texttt{Η} (OR):} Επιστρέφει \texttt{ΑΛΗΘΗΣ} αν \textbf{τουλάχιστον ένας} τελεστέος είναι \texttt{ΑΛΗΘΗΣ}.
    \item \textbf{\texttt{ΟΧΙ} (NOT):} Αντιστρέφει τη λογική τιμή (μοναδιαίος τελεστής).
\end{itemize}

\subsection{Παρατηρήσεις για τους τελεστές}

\begin{enumerate}[label=\arabic*.]
    \item \textbf{DIV και MOD με αρνητικούς:} Υπάρχει ασυμφωνία σχετικά με το αποτέλεσμα των \texttt{DIV} και \texttt{MOD} για αρνητικούς τελεστέους, καθώς και σχετική επιλογή του Διερμηνευτή για να προσαρμόσετε το αποτέλεσμα.
    
    \begin{tcolorbox}[colback=red!10,colframe=red!50!black,title=Οδηγία Παιδαγωγικού Ινστιτούτου (Φεβρουάριος 2008)]
    Δεν πρέπει να χρησιμοποιούνται σε ασκήσεις με αρνητικούς αριθμούς οι τελεστές \texttt{DIV} και \texttt{MOD}.
    \end{tcolorbox}
    
    \item \textbf{Προσεταιριστικότητα δύναμης:} Υπάρχει ασάφεια σχετικά με το αν ο τελεστής ύψωσης σε δύναμη \texttt{\^{}} εκτελείται από δεξιά προς τα αριστερά ή αντίθετα, καθώς και σχετική επιλογή στο διάλογο ρυθμίσεων του Διερμηνευτή.
    
    \item \textbf{Προτεραιότητα ΚΑΙ/Η:} Υπάρχει ασάφεια σχετικά με το αν το \texttt{ΚΑΙ} έχει μεγαλύτερη ή ίση προτεραιότητα με το \texttt{Η}, καθώς και σχετική επιλογή του Διερμηνευτή.
    
    \item \textbf{Unicode vs ANSI τελεστές:} Πριν την έκδοση 0.89 βήτα οι συγκριτικοί τελεστές μετατρεπόταν αυτόματα στους αντίστοιχους Unicode ($\leq$, $\geq$, $\neq$). Από την 0.89 βήτα και μετά υποστηρίζονται μεν οι Unicode, αλλά η \textbf{προεπιλεγμένη μορφή} των τελεστών είναι η ANSI (\texttt{>=}, \texttt{<=}, \texttt{<>}) για συμφωνία με το σχολικό βιβλίο.
\end{enumerate}

% =======================================================
\section{Τα μέρη του προγράμματος}
% =======================================================

Μέρη του προγράμματος λέμε:
\begin{enumerate}
    \item Την \textbf{επικεφαλίδα} (δηλαδή το όνομα)
    \item Το \textbf{τμήμα δήλωσης σταθερών}
    \item Το \textbf{τμήμα δήλωσης μεταβλητών}
    \item Το \textbf{κυρίως σώμα} του προγράμματος
\end{enumerate}

Στη συνέχεια βέβαια μπορεί να ακολουθούν και \textbf{υποπρογράμματα}. Τα τμήματα δηλώσεων είναι \textbf{προαιρετικά}.

\subsection{Παράδειγμα προγράμματος}

Ένα παράδειγμα προγράμματος που χρησιμοποιεί και τα τέσσερα τμήματα είναι το παρακάτω:

\begin{lstlisting}[language=,morekeywords={ΠΡΟΓΡΑΜΜΑ,ΣΤΑΘΕΡΕΣ,ΜΕΤΑΒΛΗΤΕΣ,ΠΡΑΓΜΑΤΙΚΕΣ,ΑΡΧΗ,ΓΡΑΨΕ,ΔΙΑΒΑΣΕ,ΤΕΛΟΣ_ΠΡΟΓΡΑΜΜΑΤΟΣ}]
ΠΡΟΓΡΑΜΜΑ ΕμβαδόνΚύκλου 
ΣΤΑΘΕΡΕΣ 
  π = 3.14 
ΜΕΤΑΒΛΗΤΕΣ 
  ΠΡΑΓΜΑΤΙΚΕΣ: Εμ, Ακ 
ΑΡΧΗ 
  ΓΡΑΨΕ 'Δώσε την ακτίνα' 
  ΔΙΑΒΑΣΕ Ακ 
  Εμ <- π*Ακ^2 
  ΓΡΑΨΕ 'Το εμβαδόν του κύκλου είναι ', Εμ 
ΤΕΛΟΣ_ΠΡΟΓΡΑΜΜΑΤΟΣ
\end{lstlisting}

\subsection{Δομή προγράμματος}

\begin{center}
\begin{tabular}{|l|l|l|}
\hline
\textbf{Τμήμα} & \textbf{Λέξη-κλειδί} & \textbf{Υποχρεωτικό} \\
\hline
Επικεφαλίδα & \texttt{ΠΡΟΓΡΑΜΜΑ} & Ναι \\
\hline
Σταθερές & \texttt{ΣΤΑΘΕΡΕΣ} & Όχι \\
\hline
Μεταβλητές & \texttt{ΜΕΤΑΒΛΗΤΕΣ} & Όχι \\
\hline
Κυρίως σώμα & \texttt{ΑΡΧΗ ... ΤΕΛΟΣ\_ΠΡΟΓΡΑΜΜΑΤΟΣ} & Ναι \\
\hline
\end{tabular}
\end{center}

\subsection{Παρατηρήσεις}

\begin{enumerate}[label=\arabic*.]
    \item \textbf{Αποδεκτά ονόματα αναγνωριστικών:} Τα ονόματα προγράμματος, σταθεράς ή μεταβλητής πρέπει να:
    \begin{itemize}
        \item Ξεκινούν με \textbf{γράμμα} (είτε ελληνικό είτε αγγλικό)
        \item Στη συνέχεια περιλαμβάνουν γράμματα, αριθμούς ή την κάτω παύλα (\texttt{\_})
    \end{itemize}
    
    \begin{tcolorbox}[colback=backcolour,colframe=black,title=Παραδείγματα αναγνωριστικών]
    \textbf{Έγκυρα:} \texttt{Μεταβλητή1}, \texttt{x}, \texttt{Αρ\_Μαθητών}, \texttt{counter}\\
    \textbf{Μη έγκυρα:} \texttt{1Μεταβλητή}, \texttt{Αρ-Μαθητών}, \texttt{my var}
    \end{tcolorbox}
    
    \item \textbf{Εκφράσεις στις σταθερές:} Στο τμήμα δήλωσης σταθερών επιτρέπεται και η χρήση τελεστών ή των ενσωματωμένων συναρτήσεων:
    
\begin{lstlisting}[language=,morekeywords={ΣΤΑΘΕΡΕΣ}]
ΣΤΑΘΕΡΕΣ
  Ν = 10
  στ = 2 * Τ_Ρ(Ν)
\end{lstlisting}
    
    \item \textbf{Πολλαπλές δηλώσεις τύπων:} Στο τμήμα δήλωσης μεταβλητών επιτρέπεται να χρησιμοποιούμε τους τύπους δεδομένων \textbf{όσες φορές θέλουμε}, δηλαδή μπορούμε για παράδειγμα να έχουμε δύο γραμμές με τη λέξη \texttt{ΠΡΑΓΜΑΤΙΚΕΣ}:
    
\begin{lstlisting}[language=,morekeywords={ΜΕΤΑΒΛΗΤΕΣ,ΠΡΑΓΜΑΤΙΚΕΣ,ΑΚΕΡΑΙΕΣ}]
ΜΕΤΑΒΛΗΤΕΣ
  ΠΡΑΓΜΑΤΙΚΕΣ: x, y
  ΑΚΕΡΑΙΕΣ: i, j
  ΠΡΑΓΜΑΤΙΚΕΣ: z    ! Επιτρέπεται
\end{lstlisting}
\end{enumerate}

% =======================================================
\section{Η εντολή ανάθεσης τιμής (\texttt{<-})}
% =======================================================

Η εντολή ανάθεσης τιμής μπορεί να μεταφραστεί ως εξής:

\begin{tcolorbox}[colback=blue!5,colframe=blue!50!black,title=Ερμηνεία εντολής ανάθεσης]
«Υπολόγισε την τιμή της έκφρασης \textbf{δεξιά} από το σύμβολο ανάθεσης τιμής και βάλε το αποτέλεσμα στη \textbf{μεταβλητή} που βρίσκεται \textbf{αριστερά} από την ανάθεση τιμής.»
\end{tcolorbox}

\subsection{Σύνταξη}

\begin{center}
\texttt{μεταβλητή <- έκφραση}
\end{center}

\subsection{Παράδειγμα}

\begin{lstlisting}[language=,morekeywords={ΠΡΟΓΡΑΜΜΑ,ΜΕΤΑΒΛΗΤΕΣ,ΑΚΕΡΑΙΕΣ,ΠΡΑΓΜΑΤΙΚΕΣ,ΧΑΡΑΚΤΗΡΕΣ,ΛΟΓΙΚΕΣ,ΑΡΧΗ,ΤΕΛΟΣ_ΠΡΟΓΡΑΜΜΑΤΟΣ}]
ΠΡΟΓΡΑΜΜΑ ΗΕντολήΑνάθεσηςΤιμής 
!Το πρόγραμμα αυτό δεν περιλαμβάνει «Γράψε». Για να δείτε τις 
!τιμές των μεταβλητών κοιτάξτε στην καρτέλα «Μεταβλητές». 
ΜΕΤΑΒΛΗΤΕΣ 
  ΑΚΕΡΑΙΕΣ: α 
  ΠΡΑΓΜΑΤΙΚΕΣ: π 
  ΧΑΡΑΚΤΗΡΕΣ: χ 
  ΛΟΓΙΚΕΣ: λ 
ΑΡΧΗ 
  α <- 1 
  π <- 2*α            !Επιτρέπεται κι ας είναι ακέραια η έκφραση 
  χ <- 'Κείμενο' 
  λ <- α >= π         !Το λ θα γίνει Ψευδής 
ΤΕΛΟΣ_ΠΡΟΓΡΑΜΜΑΤΟΣ
\end{lstlisting}

\subsection{Παρατηρήσεις}

\begin{enumerate}[label=\arabic*.]
    \item \textbf{Δεξί και αριστερό μέρος:} 
    \begin{itemize}
        \item Στο \textbf{δεξί μέρος} μπορεί να βρίσκεται οτιδήποτε έχει τιμή (έκφραση, σταθερά, μεταβλητή, κλήση συνάρτησης κ.λπ.)
        \item Στο \textbf{αριστερό μέρος} μπορεί να βρίσκεται μία και μόνο μεταβλητή (ή στοιχείο πίνακα) συμβατού τύπου δεδομένων με την έκφραση
    \end{itemize}
    
    \item \textbf{Συμβατότητα τύπων:} Η μόνη συμβατή ανάθεση μεταξύ διαφορετικών τύπων δεδομένων είναι από \textbf{ακέραιο} (δεξιά) σε \textbf{πραγματικό} (αριστερά).
    
    \begin{center}
    \begin{tabular}{|c|c|c|}
    \hline
    \textbf{Αριστερά (μεταβλητή)} & \textbf{Δεξιά (έκφραση)} & \textbf{Επιτρέπεται;} \\
    \hline
    Ακέραιος & Ακέραιος & \checkmark \\
    \hline
    Πραγματικός & Πραγματικός & \checkmark \\
    \hline
    Πραγματικός & Ακέραιος & \checkmark \\
    \hline
    Ακέραιος & Πραγματικός & $\times$ \\
    \hline
    Χαρακτήρες & Χαρακτήρες & \checkmark \\
    \hline
    Λογικός & Λογικός & \checkmark \\
    \hline
    \end{tabular}
    \end{center}
    
    \item \textbf{Unicode vs ANSI:} Πριν την έκδοση 0.90 βήτα ο τελεστής ανάθεσης τιμής μετατρεπόταν αυτόματα στη Unicode μορφή του ($\leftarrow$). Πλέον υποστηρίζεται και το βελάκι, αλλά η \textbf{προεπιλεγμένη μορφή} είναι η ANSI (\texttt{<-}) για συμφωνία με το σχολικό βιβλίο.
    
    \begin{tcolorbox}[colback=yellow!10,colframe=orange!50!black,title=Σημείωση]
    Αυτή η συμπεριφορά μπορεί να προσαρμοστεί από το μενού \textbf{Εργαλεία $\rightarrow$ Επιλογές} του Διερμηνευτή.
    \end{tcolorbox}
\end{enumerate}

% =======================================================
\section{Η εντολή ΓΡΑΨΕ}
% =======================================================

Η εντολή \texttt{ΓΡΑΨΕ} εμφανίζει στην οθόνη τις τιμές μιας λίστας εκφράσεων, χωρισμένων από κόμματα.

\subsection{Σύνταξη}

\begin{center}
\texttt{ΓΡΑΨΕ έκφραση1, έκφραση2, ..., έκφρασηΝ}
\end{center}

\subsection{Παράδειγμα}

\begin{lstlisting}[language=,morekeywords={ΠΡΟΓΡΑΜΜΑ,ΜΕΤΑΒΛΗΤΕΣ,ΑΚΕΡΑΙΕΣ,ΑΡΧΗ,ΓΡΑΨΕ,ΤΕΛΟΣ_ΠΡΟΓΡΑΜΜΑΤΟΣ}]
ΠΡΟΓΡΑΜΜΑ ΗΕντολήΓράψε 
!Εμφανίζει «10 3.00 κείμενο ΨΕΥΔΗΣ» 
ΜΕΤΑΒΛΗΤΕΣ 
  ΑΚΕΡΑΙΕΣ: α 
ΑΡΧΗ 
  α <- 10 
  ΓΡΑΨΕ α, ' ', Τ_Ρ(α - 1), ' κείμενο ', 1 = 3 
ΤΕΛΟΣ_ΠΡΟΓΡΑΜΜΑΤΟΣ
\end{lstlisting}

\subsection{Παρατηρήσεις}

\begin{enumerate}[label=\arabic*.]
    \item \textbf{Τύποι εκφράσεων:} Οι εκφράσεις μπορεί να είναι \textbf{οποιουδήποτε τύπου} δεδομένων. Αν είναι λογικού τύπου, στην οθόνη εμφανίζεται \texttt{ΑΛΗΘΗΣ} ή \texttt{ΨΕΥΔΗΣ}.
    
    \item \textbf{ΓΡΑΨΕ χωρίς παραμέτρους:} Επιτρέπεται εντολή \texttt{ΓΡΑΨΕ} χωρίς παραμέτρους. Απλά κατεβάζει κατά μία γραμμή το δρομέα (σελ. 180 του βιβλίου).
    
\begin{lstlisting}[language=,morekeywords={ΓΡΑΨΕ}]
ΓΡΑΨΕ           ! Αλλαγή γραμμής
\end{lstlisting}
    
    \item \textbf{Κενά μεταξύ τιμών:} Για ευκολία σε εισαγωγικές δραστηριότητες των μαθητών, μια εντολή \texttt{ΓΡΑΨΕ 1, 2, 3} αφήνει κενά μεταξύ των εμφανιζόμενων αριθμών (δηλαδή εμφανίζει \texttt{1 2 3} αντί για \texttt{123}).
    
    \begin{tcolorbox}[colback=yellow!10,colframe=orange!50!black,title=Σημείωση]
    Αυτή η συμπεριφορά μπορεί να απενεργοποιηθεί από το μενού \textbf{Εργαλεία $\rightarrow$ Επιλογές}.
    \end{tcolorbox}
    
    \item \textbf{Παραμονή δρομέα στην ίδια γραμμή:} Στη ΓΛΩΣΣΑ δεν υπάρχει εντολή αντίστοιχη της \texttt{write} της Pascal, η οποία να αφήνει το δρομέα στην ίδια γραμμή. Η λύση που υλοποιήθηκε είναι η εξής:
    
    \begin{tcolorbox}[colback=blue!5,colframe=blue!50!black,title=Κανόνας για τη θέση του δρομέα]
    Αφού αποτιμηθεί όλο το κείμενο που προκύπτει από τη \texttt{ΓΡΑΨΕ}, αν ο \textbf{τελευταίος χαρακτήρας} του είναι το κενό, τότε αυτή μεταφράζεται σε \texttt{write} (δρομέας στην ίδια γραμμή).
    \end{tcolorbox}
    
    \begin{center}
    \begin{tabular}{|l|l|}
    \hline
    \textbf{Εντολή} & \textbf{Αποτέλεσμα} \\
    \hline
    \texttt{ΓΡΑΨΕ 'Δώσε το όνομά σου:~~'} & Δρομέας στην \textbf{ίδια} γραμμή \\
    \hline
    \texttt{ΓΡΑΨΕ 'Δώσε το όνομά σου:'} & Δρομέας στην \textbf{επόμενη} γραμμή \\
    \hline
    \end{tabular}
    \end{center}
    
    \begin{tcolorbox}[colback=yellow!10,colframe=orange!50!black,title=Προσοχή]
    Επειδή μερικές φορές χρειάζεται να μην υπάρχει κενό στο τέλος μίας \texttt{ΓΡΑΨΕ}, το τελευταίο κενό \textbf{διαγράφεται} και δεν εμφανίζεται στην οθόνη εκτέλεσης. Γι' αυτό το λόγο, αν θέλετε να εμφανιστεί ένα κενό στο τέλος, πρέπει να βάλετε \textbf{δύο κενά}.
    
    Μπορείτε να απενεργοποιήσετε αυτήν τη συμπεριφορά από το μενού \textbf{Εργαλεία $\rightarrow$ Επιλογές}.
    \end{tcolorbox}
\end{enumerate}

\subsection{Παραδείγματα χρήσης}

\begin{lstlisting}[language=,morekeywords={ΓΡΑΨΕ}]
ΓΡΑΨΕ 'Καλημέρα!'                  ! Εμφάνιση κειμένου
ΓΡΑΨΕ 'x = ', x                    ! Εμφάνιση μεταβλητής
ΓΡΑΨΕ α, β, γ                      ! Εμφάνιση πολλών τιμών
ΓΡΑΨΕ 'Αποτέλεσμα: ', α > β        ! Εμφάνιση λογικής τιμής
ΓΡΑΨΕ 'Δώσε τιμή:  '               ! Δρομέας στην ίδια γραμμή
\end{lstlisting}

% =======================================================
\section{Η εντολή ΔΙΑΒΑΣΕ}
% =======================================================

Η εντολή \texttt{ΔΙΑΒΑΣΕ} διαβάζει από το πληκτρολόγιο μία λίστα μεταβλητών, χωρισμένων από κόμματα. Για να το επιτύχει αυτό, σε κάθε \texttt{ΔΙΑΒΑΣΕ} η αντίστοιχη γραμμή της καρτέλας «Οθόνη εκτέλεσης» χρωματίζεται για να προσελκύσει την προσοχή του χρήστη και η εκτέλεση του προγράμματος παύει μέχρι να εισαχθούν τα δεδομένα και να πατηθεί το \texttt{[Enter]}.

Οι μεταβλητές αυτές μπορεί να είναι οποιουδήποτε τύπου δεδομένων, \textbf{εκτός από λογικού}.

\subsection{Σύνταξη}

\begin{center}
\texttt{ΔΙΑΒΑΣΕ μεταβλητή1, μεταβλητή2, ..., μεταβλητήΝ}
\end{center}

\subsection{Παράδειγμα}

\begin{lstlisting}[language=,morekeywords={ΠΡΟΓΡΑΜΜΑ,ΜΕΤΑΒΛΗΤΕΣ,ΑΚΕΡΑΙΕΣ,ΠΡΑΓΜΑΤΙΚΕΣ,ΧΑΡΑΚΤΗΡΕΣ,ΑΡΧΗ,ΔΙΑΒΑΣΕ,ΤΕΛΟΣ_ΠΡΟΓΡΑΜΜΑΤΟΣ}]
ΠΡΟΓΡΑΜΜΑ ΗΕντολήΔΙΑΒΑΣΕ 
!Διαβάζει μία ακέραια μεταβλητή, μία πραγματική και μία 
!αλφαριθμητική. Οι λογικές δεν είναι δυνατόν να διαβαστούν. 
!Αντίθετα με τις περισσότερες γλώσσες προγραμματισμού, όταν 
!μία ΔΙΑΒΑΣΕ έχει πολλά ορίσματα, αυτά πρέπει να εισαχθούν 
!σε ξεχωριστές γραμμές (με [Enter] ανάμεσα από τα ορίσματα). 
ΜΕΤΑΒΛΗΤΕΣ 
  ΑΚΕΡΑΙΕΣ: α 
  ΠΡΑΓΜΑΤΙΚΕΣ: Π[10] 
  ΧΑΡΑΚΤΗΡΕΣ: κ 
ΑΡΧΗ 
  ΔΙΑΒΑΣΕ α, Π[23 DIV 3]    !διαβάζει το α και το Π[7] 
  ΔΙΑΒΑΣΕ κ 
ΤΕΛΟΣ_ΠΡΟΓΡΑΜΜΑΤΟΣ
\end{lstlisting}

\subsection{Παρατηρήσεις}

\begin{enumerate}[label=\arabic*.]
    \item \textbf{Μόνο μεταβλητές:} Δεν επιτρέπεται να διαβαστεί κάτι που δεν είναι μεταβλητή. Για παράδειγμα, η εντολή:
    
\begin{lstlisting}[language=,morekeywords={ΔΙΑΒΑΣΕ}]
ΔΙΑΒΑΣΕ α + β, -π, 24, 'κείμενο'    ! ΛΑΘΟΣ!
\end{lstlisting}
    
    είναι τελείως λάθος.
    
    \item \textbf{Όχι λογικές μεταβλητές:} Δεν είναι δυνατό να διαβαστούν λογικές μεταβλητές. Αντί αυτού μπορούμε να διαβάσουμε έναν ακέραιο ή ένα χαρακτήρα και στη συνέχεια να δώσουμε την αντίστοιχη τιμή στη λογική μεταβλητή με μια εντολή \texttt{ΑΝ}:
    
\begin{lstlisting}[language=,morekeywords={ΠΡΟΓΡΑΜΜΑ,ΜΕΤΑΒΛΗΤΕΣ,ΑΚΕΡΑΙΕΣ,ΛΟΓΙΚΕΣ,ΑΡΧΗ,ΓΡΑΨΕ,ΔΙΑΒΑΣΕ,ΑΝ,ΤΟΤΕ,ΑΛΛΙΩΣ,ΤΕΛΟΣ_ΑΝ,ΤΕΛΟΣ_ΠΡΟΓΡΑΜΜΑΤΟΣ,ΑΛΗΘΗΣ,ΨΕΥΔΗΣ}]
ΠΡΟΓΡΑΜΜΑ ΔιάβασμαΛογικήςΜεταβλητής 
ΜΕΤΑΒΛΗΤΕΣ 
  ΑΚΕΡΑΙΕΣ: ι 
  ΛΟΓΙΚΕΣ: φύλο 
ΑΡΧΗ 
  ΓΡΑΨΕ 'Δώσε 0 αν είσαι γυναίκα και 1 αν είσαι άντρας: ' 
  ΔΙΑΒΑΣΕ ι 
  ΑΝ ι = 0 ΤΟΤΕ 
    φύλο <- ΑΛΗΘΗΣ 
  ΑΛΛΙΩΣ 
    φύλο <- ΨΕΥΔΗΣ 
  ΤΕΛΟΣ_ΑΝ 
  !Η παραπάνω εντολή Αν ισοδυναμεί με την εντολή «φύλο <- ι = 0» 
ΤΕΛΟΣ_ΠΡΟΓΡΑΜΜΑΤΟΣ
\end{lstlisting}
    
    \item \textbf{Έλεγχος δεδομένων κατά την είσοδο:} Στο Διερμηνευτή υπάρχει η επιλογή να ελέγχονται τα δεδομένα κατά την είσοδο. Αυτό σημαίνει ότι αν το πρόγραμμα προσπαθεί να διαβάσει:
    \begin{itemize}
        \item έναν \textbf{ακέραιο}: η περιοχή εισόδου επιτρέπει μόνο νούμερα
        \item έναν \textbf{πραγματικό}: επιτρέπει νούμερα και μία τελεία
    \end{itemize}
    
    \item \textbf{Ξεχωριστές γραμμές για κάθε τιμή:} Σε εντολές του τύπου \texttt{ΔΙΑΒΑΣΕ χ, ψ, ω} \textbf{δεν επιτρέπεται} να εισαχθούν όλα τα δεδομένα σε μία γραμμή, αλλά πρέπει κάθε τιμή να εισάγεται χωριστά πατώντας ισάριθμες φορές το \texttt{[Enter]} (αντίθετα από τις περισσότερες γλώσσες προγραμματισμού).
    
    \begin{tcolorbox}[colback=blue!5,colframe=blue!50!black,title=Γιατί αυτή η συμπεριφορά;]
    Αυτό γίνεται για ευκολία των μαθητών, και για να μη δημιουργηθούν προβλήματα: για παράδειγμα η παράλληλη χρήση των \texttt{read} και \texttt{readln} της Pascal μπερδεύει εύκολα ακόμα και πεπειραμένους προγραμματιστές, ενώ στην C η \texttt{scanf} δε μπορεί να διαβάσει αλφαριθμητικά που να περιέχουν κενά.
    \end{tcolorbox}
    
    \begin{tcolorbox}[colback=yellow!10,colframe=orange!50!black,title=Παράδειγμα προβλήματος]
    Σε μια εντολή \texttt{ΔΙΑΒΑΣΕ όνομα, ηλικία} αν ο χρήστης έδινε «Άλκης Γεωργόπουλος 35» πού θα έπρεπε να σταματήσει το διάβασμα του αλφαριθμητικού;
    \end{tcolorbox}
\end{enumerate}

\subsection{Συνοπτικός πίνακας τύπων που διαβάζονται}

\begin{center}
\begin{tabular}{|l|c|l|}
\hline
\textbf{Τύπος} & \textbf{Διαβάζεται;} & \textbf{Σημείωση} \\
\hline
Ακέραιος & \checkmark & Μόνο αριθμοί \\
\hline
Πραγματικός & \checkmark & Αριθμοί και τελεία \\
\hline
Χαρακτήρες & \checkmark & Οτιδήποτε \\
\hline
Λογικός & $\times$ & Χρήση ΑΝ για μετατροπή \\
\hline
\end{tabular}
\end{center}

% =======================================================
\section{Η εντολή ΑΝ}
% =======================================================

Η εντολή \texttt{ΑΝ} χρησιμοποιείται όταν χρειάζεται \textbf{διακλάδωση} της ροής του προγράμματος, ανάλογα με κάποια συνθήκη.

\subsection{Σύνταξη}

\begin{center}
\begin{tabular}{|l|}
\hline
\texttt{ΑΝ συνθήκη ΤΟΤΕ} \\
\texttt{~~~~εντολές} \\
\texttt{ΑΛΛΙΩΣ\_ΑΝ συνθήκη2 ΤΟΤΕ} \\
\texttt{~~~~εντολές} \\
\texttt{ΑΛΛΙΩΣ} \\
\texttt{~~~~εντολές} \\
\texttt{ΤΕΛΟΣ\_ΑΝ} \\
\hline
\end{tabular}
\end{center}

\begin{tcolorbox}[colback=yellow!10,colframe=orange!50!black,title=Σημείωση]
Τα τμήματα \texttt{ΑΛΛΙΩΣ\_ΑΝ} και \texttt{ΑΛΛΙΩΣ} είναι \textbf{προαιρετικά}. Μπορούν να υπάρχουν πολλαπλά \texttt{ΑΛΛΙΩΣ\_ΑΝ}, αλλά το πολύ ένα \texttt{ΑΛΛΙΩΣ}.
\end{tcolorbox}

\subsection{Παράδειγμα}

\begin{lstlisting}[language=,morekeywords={ΠΡΟΓΡΑΜΜΑ,ΜΕΤΑΒΛΗΤΕΣ,ΠΡΑΓΜΑΤΙΚΕΣ,ΑΡΧΗ,ΓΡΑΨΕ,ΔΙΑΒΑΣΕ,ΑΝ,ΤΟΤΕ,ΑΛΛΙΩΣ_ΑΝ,ΑΛΛΙΩΣ,ΤΕΛΟΣ_ΑΝ,ΤΕΛΟΣ_ΠΡΟΓΡΑΜΜΑΤΟΣ}]
ΠΡΟΓΡΑΜΜΑ ΗΕντολήΑν 
!Διαβάζει ένα βαθμό και εμφανίζει 
!στο χρήστη αν πέρασε ή αν κόπηκε. 
ΜΕΤΑΒΛΗΤΕΣ 
  ΠΡΑΓΜΑΤΙΚΕΣ: βαθμός 
ΑΡΧΗ 
  ΓΡΑΨΕ 'Δώσε βαθμό:  ' 
  ΔΙΑΒΑΣΕ βαθμός 
  ΑΝ βαθμός < 0 ΤΟΤΕ 
    ΓΡΑΨΕ 'Δεν υπάρχει αρνητικός βαθμός' 
  ΑΛΛΙΩΣ_ΑΝ βαθμός < 10 ΤΟΤΕ 
    ΓΡΑΨΕ 'Κόπηκες' 
  ΑΛΛΙΩΣ_ΑΝ βαθμός <= 20 ΤΟΤΕ 
    ΓΡΑΨΕ 'Πέρασες' 
  ΑΛΛΙΩΣ 
    ΓΡΑΨΕ 'Ο βαθμός πρέπει να είναι στην εικοσαβάθμια,', 
    & ' όχι στην εκατοσταβάθμια κλίμακα' 
  ΤΕΛΟΣ_ΑΝ 
ΤΕΛΟΣ_ΠΡΟΓΡΑΜΜΑΤΟΣ
\end{lstlisting}

\subsection{Παρατηρήσεις}

\begin{enumerate}[label=\arabic*.]
    \item \textbf{ΑΛΛΙΩΣ\_ΑΝ vs ΑΛΛΙΩΣ ΑΝ:} Η εντολή \texttt{ΑΛΛΙΩΣ\_ΑΝ} είναι \textbf{διαφορετική} από την \texttt{ΑΛΛΙΩΣ ΑΝ}.
    
    \begin{itemize}
        \item Η \texttt{ΑΛΛΙΩΣ\_ΑΝ} είναι \textbf{μία εντολή} (else-if)
        \item Η \texttt{ΑΛΛΙΩΣ ΑΝ} είναι \textbf{δύο εντολές}: το μέρος \texttt{ΑΛΛΙΩΣ} μιας εντολής \texttt{ΑΝ} και μία δεύτερη \texttt{ΑΝ} στη συνέχεια
    \end{itemize}
    
    \begin{tcolorbox}[colback=red!10,colframe=red!50!black,title=Προσοχή]
    Στην περίπτωση \texttt{ΑΛΛΙΩΣ ΑΝ}, η δεύτερη \texttt{ΑΝ} \textbf{δεν επιτρέπεται} να βρίσκεται στην ίδια γραμμή με το \texttt{ΑΛΛΙΩΣ} --- πρέπει να γραφεί στην επόμενη γραμμή.
    \end{tcolorbox}
    
    \item \textbf{Αριθμός ΤΕΛΟΣ\_ΑΝ:}
    \begin{itemize}
        \item Με \texttt{ΑΛΛΙΩΣ\_ΑΝ}: χρειάζεται \textbf{ένα} \texttt{ΤΕΛΟΣ\_ΑΝ}
        \item Με \texttt{ΑΛΛΙΩΣ ΑΝ}: χρειάζονται \textbf{τόσα} \texttt{ΤΕΛΟΣ\_ΑΝ} όσα και τα \texttt{ΑΝ}
    \end{itemize}
    
    \begin{tcolorbox}[colback=backcolour,colframe=black,title=Σύγκριση δομών]
    \textbf{Με ΑΛΛΙΩΣ\_ΑΝ (1 ΤΕΛΟΣ\_ΑΝ):}
\begin{lstlisting}[language=,morekeywords={ΑΝ,ΤΟΤΕ,ΑΛΛΙΩΣ_ΑΝ,ΑΛΛΙΩΣ,ΤΕΛΟΣ_ΑΝ},numbers=none,frame=none,backgroundcolor=\color{white}]
ΑΝ x < 0 ΤΟΤΕ
  ...
ΑΛΛΙΩΣ_ΑΝ x < 10 ΤΟΤΕ
  ...
ΤΕΛΟΣ_ΑΝ
\end{lstlisting}
    
    \textbf{Με ΑΛΛΙΩΣ ΑΝ (2 ΤΕΛΟΣ\_ΑΝ):}
\begin{lstlisting}[language=,morekeywords={ΑΝ,ΤΟΤΕ,ΑΛΛΙΩΣ,ΤΕΛΟΣ_ΑΝ},numbers=none,frame=none,backgroundcolor=\color{white}]
ΑΝ x < 0 ΤΟΤΕ
  ...
ΑΛΛΙΩΣ
  ΑΝ x < 10 ΤΟΤΕ
    ...
  ΤΕΛΟΣ_ΑΝ
ΤΕΛΟΣ_ΑΝ
\end{lstlisting}
    \end{tcolorbox}
\end{enumerate}

% =======================================================
\section{Η εντολή ΕΠΙΛΕΞΕ}
% =======================================================

Η εντολή \texttt{ΕΠΙΛΕΞΕ} χρησιμοποιείται αντί για πολλές \texttt{ΑΝ ... ΑΛΛΙΩΣ\_ΑΝ ...} όταν θέλουμε να πάρουμε περιπτώσεις ανάλογα με την τιμή \textbf{μίας μόνο έκφρασης}. Δηλαδή δεν μπορεί να αντικαταστήσει πολλές \texttt{ΑΝ} οι οποίες αναφέρονται σε διαφορετικές συνθήκες.

\subsection{Σύνταξη}

\begin{center}
\begin{tabular}{|l|}
\hline
\texttt{ΕΠΙΛΕΞΕ έκφραση} \\
\texttt{~~~~ΠΕΡΙΠΤΩΣΗ τιμή1} \\
\texttt{~~~~~~~~εντολές} \\
\texttt{~~~~ΠΕΡΙΠΤΩΣΗ τιμή2, τιμή3..τιμή4} \\
\texttt{~~~~~~~~εντολές} \\
\texttt{~~~~ΠΕΡΙΠΤΩΣΗ ΑΛΛΙΩΣ} \\
\texttt{~~~~~~~~εντολές} \\
\texttt{ΤΕΛΟΣ\_ΕΠΙΛΟΓΩΝ} \\
\hline
\end{tabular}
\end{center}

\subsection{Παράδειγμα}

\begin{lstlisting}[language=,morekeywords={ΠΡΟΓΡΑΜΜΑ,ΜΕΤΑΒΛΗΤΕΣ,ΠΡΑΓΜΑΤΙΚΕΣ,ΑΡΧΗ,ΓΡΑΨΕ,ΔΙΑΒΑΣΕ,ΕΠΙΛΕΞΕ,ΠΕΡΙΠΤΩΣΗ,ΑΛΛΙΩΣ,ΤΕΛΟΣ_ΕΠΙΛΟΓΩΝ,ΤΕΛΟΣ_ΠΡΟΓΡΑΜΜΑΤΟΣ}]
ΠΡΟΓΡΑΜΜΑ ΗΕντολήΕΠΙΛΕΞΕ 
ΜΕΤΑΒΛΗΤΕΣ 
  ΠΡΑΓΜΑΤΙΚΕΣ: βαθμός 
ΑΡΧΗ 
  ΓΡΑΨΕ 'Δώσε βαθμό:  ' 
  ΔΙΑΒΑΣΕ βαθμός 
  ΕΠΙΛΕΞΕ βαθμός 
    ΠΕΡΙΠΤΩΣΗ 15..20                         !15 <= βαθμός <= 20 
      ΓΡΑΨΕ 'Είσαι άξιος' 
    ΠΕΡΙΠΤΩΣΗ 10..15                          !10 <= βαθμός < 15 
      ΓΡΑΨΕ 'Καλούτσικα πήγες' 
    ΠΕΡΙΠΤΩΣΗ 0                                      !βαθμός = 0 
      ΓΡΑΨΕ 'Δεν προσήλθες στις εξετάσεις' 
    ΠΕΡΙΠΤΩΣΗ 0..10                             !0 < βαθμός < 10 
      ΓΡΑΨΕ 'Κόπηκες' 
    ΠΕΡΙΠΤΩΣΗ > 20 
      ΓΡΑΨΕ 'Ο βαθμός θα πρέπει να είναι στην εικοσαβάθμια, ', 
      & 'όχι στην εκατοσταβάθμια κλίμακα' 
    ΠΕΡΙΠΤΩΣΗ ΑΛΛΙΩΣ 
      ΓΡΑΨΕ 'Έδωσες άκυρο βαθμό' 
  ΤΕΛΟΣ_ΕΠΙΛΟΓΩΝ 
ΤΕΛΟΣ_ΠΡΟΓΡΑΜΜΑΤΟΣ
\end{lstlisting}

\subsection{Παρατηρήσεις}

\begin{enumerate}[label=\arabic*.]
    \item \textbf{Τύπος έκφρασης:} Η έκφραση στην \texttt{ΕΠΙΛΕΞΕ} μπορεί να είναι \textbf{οποιουδήποτε τύπου} δεδομένων.
    
    \begin{tcolorbox}[colback=blue!5,colframe=blue!50!black,title=Παραδείγματα από το σχολικό εγχειρίδιο]
    \begin{itemize}
        \item Με \textbf{πραγματικούς αριθμούς}: σελίδα 75 του τετραδίου μαθητή
        \item Με \textbf{αλφαριθμητικά}: σελίδα 269 του βιβλίου μαθητή
    \end{itemize}
    \end{tcolorbox}
    
    \item \textbf{Συγκριτικοί τελεστές στις περιπτώσεις:} Επιτρέπεται η χρήση συγκριτικών τελεστών στις περιπτώσεις της \texttt{ΕΠΙΛΕΞΕ}, για παράδειγμα \texttt{ΠΕΡΙΠΤΩΣΗ > 20}. Σχετικό παράδειγμα υπάρχει στη σελίδα 75 του τετραδίου μαθητή.
    
    \begin{tcolorbox}[colback=red!10,colframe=red!50!black,title=Προσοχή]
    \textbf{Δεν} επιτρέπεται η χρήση \textbf{λογικών τελεστών} στις περιπτώσεις. Δηλαδή η εντολή:
    
    \texttt{ΠΕΡΙΠΤΩΣΗ > 20 ΚΑΙ < 40}
    
    \textbf{δεν είναι αποδεκτή}.
    \end{tcolorbox}
    
    \item \textbf{Σειρά περιπτώσεων με πραγματικούς:} Προσέξτε τη σειρά των περιπτώσεων όταν έχουμε πραγματικούς αριθμούς!
    
    Στο παράδειγμα, ο βαθμός θεωρείται πραγματικός, οπότε μια τιμή όπως 9.5 είναι αποδεκτή και πρέπει να εμφανίσει «Κόπηκες». Αν όμως η περίπτωση \texttt{0..10} προηγούταν της \texttt{10..15}, τότε το 10 θα πήγαινε στην πρώτη περίπτωση και το αποτέλεσμα θα ήταν λάθος.
    
    \begin{tcolorbox}[colback=yellow!10,colframe=orange!50!black,title=Κανόνας για πραγματικούς]
    Στην \texttt{ΕΠΙΛΕΞΕ}, όταν έχουμε να κάνουμε με \textbf{πραγματικούς}, πρέπει να προηγούνται τα \textbf{κλειστά διαστήματα} (π.χ. $[10, 15)$) και μετά τα \textbf{ανοιχτά} (π.χ. $(0, 10)$).
    \end{tcolorbox}
    
    \item \textbf{Πολλαπλές εκφράσεις ανά περίπτωση:} Η κάθε \texttt{ΠΕΡΙΠΤΩΣΗ} μπορεί να περιέχει πολλές εκφράσεις χωρισμένες με κόμματα:
    
\begin{lstlisting}[language=,morekeywords={ΠΕΡΙΠΤΩΣΗ},numbers=none]
ΠΕΡΙΠΤΩΣΗ Τ_Ρ(Ν/2) .. α, β + 25, 10 .. 9
\end{lstlisting}
    
    Η παραπάνω περίπτωση ισοδυναμεί με την εντολή:
    
\begin{lstlisting}[language=,morekeywords={ΑΝ,ΚΑΙ,Η,ΤΟΤΕ},numbers=none]
ΑΝ (επιλογέας >= Τ_Ρ(Ν/2) ΚΑΙ επιλογέας <= α) Η
   (επιλογέας = β + 25) Η
   (επιλογέας >= 10 ΚΑΙ επιλογέας <= 9) ΤΟΤΕ ...
\end{lstlisting}
    
    \begin{tcolorbox}[colback=backcolour,colframe=black,title=Ανάλυση]
    \begin{itemize}
        \item \texttt{Τ\_Ρ(Ν/2) .. α}: διάστημα τιμών
        \item \texttt{β + 25}: συγκεκριμένη τιμή (όχι διάστημα)
        \item \texttt{10 .. 9}: κενό διάστημα (το κάτω άκρο > πάνω άκρο)
    \end{itemize}
    \end{tcolorbox}
    
    \item \textbf{ΕΠΙΛΕΞΕ χωρίς περιπτώσεις:} Από τον Διερμηνευτή της ΓΛΩΣΣΑΣ:
    \begin{itemize}
        \item \textbf{Αποδεκτή}: \texttt{ΕΠΙΛΕΞΕ} που περιέχει μόνο \texttt{ΠΕΡΙΠΤΩΣΗ ΑΛΛΙΩΣ}
        \item \textbf{Μη αποδεκτή}: \texttt{ΕΠΙΛΕΞΕ} χωρίς καμία \texttt{ΠΕΡΙΠΤΩΣΗ}
    \end{itemize}
    
    \item \textbf{Αποτίμηση έκφρασης:} Η έκφραση στην \texttt{ΕΠΙΛΕΞΕ} αποτιμάται \textbf{μία φορά}, όχι διαδοχικά σε κάθε περίπτωση.
    
    \begin{tcolorbox}[colback=blue!5,colframe=blue!50!black,title=Σημασία για συναρτήσεις]
    Αν η έκφραση περιέχει κλήση σε συνάρτηση, η συνάρτηση θα κληθεί \textbf{μόνο μία φορά} κατά την \texttt{ΕΠΙΛΕΞΕ} και όχι σε κάθε \texttt{ΠΕΡΙΠΤΩΣΗ}.
    \end{tcolorbox}
\end{enumerate}

\subsection{Σύγκριση ΑΝ και ΕΠΙΛΕΞΕ}

\begin{center}
\begin{tabular}{|l|c|c|}
\hline
\textbf{Χαρακτηριστικό} & \textbf{ΑΝ} & \textbf{ΕΠΙΛΕΞΕ} \\
\hline
Διαφορετικές συνθήκες & \checkmark & $\times$ \\
\hline
Μία έκφραση, πολλές τιμές & \checkmark & \checkmark \\
\hline
Διαστήματα τιμών & Με λογικούς τελεστές & Με \texttt{..} \\
\hline
Συγκριτικοί τελεστές & \checkmark & \checkmark \\
\hline
Λογικοί τελεστές & \checkmark & $\times$ \\
\hline
\end{tabular}
\end{center}

% =======================================================
\section{Η εντολή ΓΙΑ}
% =======================================================

Η εντολή \texttt{ΓΙΑ} είναι η πιο απλή εντολή επανάληψης. Χρησιμοποιεί ένα \textbf{μετρητή} για να μετράει πόσες επαναλήψεις γίνονται. Επιλέγουμε να τη χρησιμοποιήσουμε όταν \textbf{γνωρίζουμε από πριν} πόσες επαναλήψεις θέλουμε.

\subsection{Σύνταξη}

\begin{center}
\begin{tabular}{|l|}
\hline
\texttt{ΓΙΑ μετρητής ΑΠΟ αρχή ΜΕΧΡΙ τέλος ΜΕ\_ΒΗΜΑ βήμα} \\
\texttt{~~~~εντολές} \\
\texttt{ΤΕΛΟΣ\_ΕΠΑΝΑΛΗΨΗΣ} \\
\hline
\end{tabular}
\end{center}

\begin{tcolorbox}[colback=yellow!10,colframe=orange!50!black,title=Σημείωση]
Αν το \texttt{ΜΕ\_ΒΗΜΑ} παραληφθεί, το βήμα θεωρείται ίσο με \textbf{1}.
\end{tcolorbox}

\subsection{Παράδειγμα}

\begin{lstlisting}[language=,morekeywords={ΠΡΟΓΡΑΜΜΑ,ΜΕΤΑΒΛΗΤΕΣ,ΠΡΑΓΜΑΤΙΚΕΣ,ΑΡΧΗ,ΓΙΑ,ΑΠΟ,ΜΕΧΡΙ,ΜΕ_ΒΗΜΑ,ΓΡΑΨΕ,ΤΕΛΟΣ_ΕΠΑΝΑΛΗΨΗΣ,ΤΕΛΟΣ_ΠΡΟΓΡΑΜΜΑΤΟΣ}]
ΠΡΟΓΡΑΜΜΑ ΗΕντολήΓια 
ΜΕΤΑΒΛΗΤΕΣ 
  ΠΡΑΓΜΑΤΙΚΕΣ: π 
ΑΡΧΗ 
  ΓΙΑ π ΑΠΟ 10 ΜΕΧΡΙ 1 ΜΕ_ΒΗΜΑ -0.5 
    ΓΡΑΨΕ π 
  ΤΕΛΟΣ_ΕΠΑΝΑΛΗΨΗΣ 
ΤΕΛΟΣ_ΠΡΟΓΡΑΜΜΑΤΟΣ
\end{lstlisting}

\subsection{Παρατηρήσεις}

\begin{enumerate}[label=\arabic*.]
    \item \textbf{ΜΕ\_ΒΗΜΑ vs ΜΕ ΒΗΜΑ:} Στο βιβλίο μαθητή το μέρος \texttt{ΜΕ ΒΗΜΑ} της εντολής \texttt{ΓΙΑ} αναφέρεται τέσσερις φορές \textbf{χωρίς κάτω παύλα}, ενώ σε όλες τις υπόλοιπες αναφορές των σχολικών εγχειριδίων είναι \textbf{με κάτω παύλα}.
    
    \begin{tcolorbox}[colback=blue!5,colframe=blue!50!black,title=Συμπεριφορά Διερμηνευτή]
    Ο Διερμηνευτής θεωρεί και τα δύο αποδεκτά, εκτός εάν αλλάξετε τη σχετική επιλογή από το μενού \textbf{Εργαλεία $\rightarrow$ Επιλογές}.
    \end{tcolorbox}
    
    \item \textbf{Τύποι εκφράσεων:} Στα μέρη \texttt{ΑΠΟ}, \texttt{ΜΕΧΡΙ} και \texttt{ΜΕ\_ΒΗΜΑ} μπορούν να χρησιμοποιηθούν \textbf{ακέραιες ή πραγματικές} εκφράσεις.
    
    \begin{tcolorbox}[colback=red!10,colframe=red!50!black,title=Προσοχή]
    Αν ο μετρητής είναι \textbf{ακέραιος}, δεν επιτρέπεται πραγματική τιμή στο \texttt{ΑΠΟ} ή στο \texttt{ΜΕ\_ΒΗΜΑ}.
    \end{tcolorbox}
    
    \item \textbf{Αποτίμηση εκφράσεων:} Οι εκφράσεις στα \texttt{ΑΠΟ}, \texttt{ΜΕΧΡΙ} και \texttt{ΜΕ\_ΒΗΜΑ} αποτιμούνται \textbf{μία μόνο φορά}. Αν περιέχουν κλήση συνάρτησης, αυτή θα κληθεί μία φορά.
    
    \item \textbf{Τιμή μετρητή μετά το τέλος:} Μετά το τέλος της επανάληψης, ο μετρητής έχει τιμή \textbf{μία θέση πέρα} από την τελική τιμή.
    
    \begin{tcolorbox}[colback=backcolour,colframe=black,title=Παράδειγμα]
    Μετά από \texttt{ΓΙΑ i ΑΠΟ 1 ΜΕΧΡΙ 10}, το \texttt{i} είναι \textbf{11} (όχι 10).
    \end{tcolorbox}
    
    \item \textbf{Όταν δεν γίνεται καμία επανάληψη:} Αν δεν πρόκειται να γίνει καμία επανάληψη (π.χ. \texttt{ΓΙΑ i ΑΠΟ 1 ΜΕΧΡΙ 0}), ο μετρητής \textbf{παίρνει την αρχική τιμή} έτσι κι αλλιώς.
\end{enumerate}

\subsection{Ισοδυναμία με ΟΣΟ}

Η μετάφραση της \texttt{ΓΙΑ} με χρήση της \texttt{ΟΣΟ} είναι:

\begin{lstlisting}[language=,morekeywords={ΟΣΟ,ΕΠΑΝΑΛΑΒΕ,ΤΕΛΟΣ_ΕΠΑΝΑΛΗΨΗΣ,Η},numbers=none]
μετρητής <- έκφραση_αρχής
τιμή_τέλους <- έκφραση_τέλους      ! Για να μη γίνεται πολλές
τιμή_βήματος <- έκφραση_βήματος   ! φορές η αποτίμηση
ΟΣΟ (μετρητής >= ή <= τιμή_τέλους) ΕΠΑΝΑΛΑΒΕ
  εντολές
  μετρητής <- μετρητής + τιμή_βήματος
ΤΕΛΟΣ_ΕΠΑΝΑΛΗΨΗΣ
\end{lstlisting}

\begin{tcolorbox}[colback=yellow!10,colframe=orange!50!black,title=Επιλογή τελεστή σύγκρισης]
\begin{itemize}
    \item Αν \texttt{τιμή\_βήματος > 0}: χρησιμοποιούμε \texttt{<=} (μικρότερο ή ίσο)
    \item Αν \texttt{τιμή\_βήματος < 0}: χρησιμοποιούμε \texttt{>=} (μεγαλύτερο ή ίσο)
\end{itemize}
\end{tcolorbox}

\subsection{Διαφορές με άλλες γλώσσες}

Η \texttt{ΓΙΑ} είναι η εντολή με τις πιο μεγάλες διαφορές στην υλοποίησή της στις γλώσσες προγραμματισμού:

\begin{center}
\begin{tabular}{|l|c|c|c|}
\hline
\textbf{Χαρακτηριστικό} & \textbf{ΓΛΩΣΣΑ} & \textbf{Pascal} & \textbf{Basic} \\
\hline
Βήμα διαφορετικό από 1 & \checkmark & $\times$ (μόνο -1) & \checkmark \\
\hline
Τιμή μετρητή μετά \texttt{for i := 1 to 10} & 11 & 10 & 11 \\
\hline
Αρχική τιμή αν 0 επαναλήψεις & Ναι & Όχι & Ναι \\
\hline
\end{tabular}
\end{center}

% =======================================================
\section{Η εντολή ΟΣΟ}
% =======================================================

Η εντολή \texttt{ΟΣΟ} είναι η πιο \textbf{ισχυρή} εντολή επανάληψης. Χρησιμοποιείται συνήθως όταν \textbf{δεν ξέρουμε} τον αριθμό των επαναλήψεων, αλλά οι επαναλήψεις εξαρτώνται από κάποια συνθήκη.

\begin{tcolorbox}[colback=blue!5,colframe=blue!50!black,title=Πότε επιλέγουμε την ΟΣΟ]
Την προτιμούμε από την \texttt{ΑΡΧΗ\_ΕΠΑΝΑΛΗΨΗΣ} όταν είναι πιθανό να \textbf{μη γίνει καμία επανάληψη}, γιατί στην \texttt{ΑΡΧΗ\_ΕΠΑΝΑΛΗΨΗΣ} γίνεται πάντα τουλάχιστον μία.
\end{tcolorbox}

\subsection{Σύνταξη}

\begin{center}
\begin{tabular}{|l|}
\hline
\texttt{ΟΣΟ συνθήκη ΕΠΑΝΑΛΑΒΕ} \\
\texttt{~~~~εντολές} \\
\texttt{ΤΕΛΟΣ\_ΕΠΑΝΑΛΗΨΗΣ} \\
\hline
\end{tabular}
\end{center}

\subsection{Παράδειγμα}

\begin{lstlisting}[language=,morekeywords={ΠΡΟΓΡΑΜΜΑ,ΜΕΤΑΒΛΗΤΕΣ,ΑΚΕΡΑΙΕΣ,ΑΡΧΗ,ΔΙΑΒΑΣΕ,ΟΣΟ,ΕΠΑΝΑΛΑΒΕ,ΓΡΑΨΕ,ΤΕΛΟΣ_ΕΠΑΝΑΛΗΨΗΣ,ΤΕΛΟΣ_ΠΡΟΓΡΑΜΜΑΤΟΣ,DIV}]
ΠΡΟΓΡΑΜΜΑ ΗΕντολήΌσο 
!Μετράει (με χαζό τρόπο) πόσες φορές μπορεί 
!να διαιρεθεί διαδοχικά ένας αριθμός με το 2. 
ΜΕΤΑΒΛΗΤΕΣ 
  ΑΚΕΡΑΙΕΣ: ι, α 
ΑΡΧΗ 
  ΔΙΑΒΑΣΕ α 
  ι <- 0 
  ΟΣΟ α >= 2 ΕΠΑΝΑΛΑΒΕ 
    α <- α DIV 2 
    ι <- ι + 1 
  ΤΕΛΟΣ_ΕΠΑΝΑΛΗΨΗΣ 
  ΓΡΑΨΕ 'Ο αριθμός μπορεί να διαιρεθεί ', ι, ' φορές με το 2.' 
ΤΕΛΟΣ_ΠΡΟΓΡΑΜΜΑΤΟΣ
\end{lstlisting}

\subsection{Παρατηρήσεις}

\begin{enumerate}[label=\arabic*.]
    \item \textbf{Έλεγχος συνθήκης:} Η συνθήκη ελέγχεται \textbf{πριν} από κάθε επανάληψη. Αν είναι \texttt{ΨΕΥΔΗΣ} από την αρχή, δεν εκτελείται καμία επανάληψη.
    
    \item \textbf{Χωρίς ενσωματωμένο μετρητή:} Η \texttt{ΟΣΟ} δεν περιλαμβάνει μετρητή όπως η \texttt{ΓΙΑ}.
\end{enumerate}

\subsection{Χρήση μετρητή στην ΟΣΟ}

Αν θέλετε να βάλετε μετρητή στην \texttt{ΟΣΟ}, θυμηθείτε τα \textbf{τρία βήματα} του μετρητή:

\begin{center}
\begin{tabular}{|c|l|l|}
\hline
\textbf{Βήμα} & \textbf{Περιγραφή} & \textbf{Παράδειγμα} \\
\hline
1 & Αρχικοποίηση \textbf{πριν} την \texttt{ΟΣΟ} & \texttt{ι <- 1} \\
\hline
2 & Μεταβολή \textbf{πριν} το \texttt{ΤΕΛΟΣ\_ΕΠΑΝΑΛΗΨΗΣ} & \texttt{ι <- ι + 1} \\
\hline
3 & Συνθήκη τέλους \textbf{στην} \texttt{ΟΣΟ} & \texttt{ΟΣΟ ι <= 10} \\
\hline
\end{tabular}
\end{center}

\begin{lstlisting}[language=,morekeywords={ΟΣΟ,ΕΠΑΝΑΛΑΒΕ,ΤΕΛΟΣ_ΕΠΑΝΑΛΗΨΗΣ},numbers=none]
ι <- 1                        ! 1. Αρχικοποίηση
ΟΣΟ ι <= 10 ΕΠΑΝΑΛΑΒΕ         ! 3. Συνθήκη τέλους
  ! εντολές...
  ι <- ι + 1                  ! 2. Μεταβολή
ΤΕΛΟΣ_ΕΠΑΝΑΛΗΨΗΣ
\end{lstlisting}

\subsection{Σύγκριση εντολών επανάληψης}

\begin{center}
\begin{tabular}{|l|c|c|c|}
\hline
\textbf{Χαρακτηριστικό} & \textbf{ΓΙΑ} & \textbf{ΟΣΟ} & \textbf{ΑΡΧΗ\_ΕΠΑΝ.} \\
\hline
Γνωστός αριθμός επαναλήψεων & \checkmark & & \\
\hline
Ενσωματωμένος μετρητής & \checkmark & $\times$ & $\times$ \\
\hline
Έλεγχος στην αρχή & \checkmark & \checkmark & $\times$ \\
\hline
Έλεγχος στο τέλος & $\times$ & $\times$ & \checkmark \\
\hline
Ελάχιστες επαναλήψεις & 0 & 0 & 1 \\
\hline
\end{tabular}
\end{center}

% =======================================================
\section{Η εντολή ΑΡΧΗ\_ΕΠΑΝΑΛΗΨΗΣ}
% =======================================================

Η εντολή \texttt{ΑΡΧΗ\_ΕΠΑΝΑΛΗΨΗΣ} χρησιμοποιείται όταν δεν υπάρχει συγκεκριμένος αριθμός επαναλήψεων, αλλά οι επαναλήψεις εξαρτώνται από κάποια συνθήκη. Είναι η αντίστοιχη της \texttt{repeat - until} της Pascal και της \texttt{do - while} στην Basic και στη C.

\begin{tcolorbox}[colback=blue!5,colframe=blue!50!black,title=Πότε επιλέγουμε την ΑΡΧΗ\_ΕΠΑΝΑΛΗΨΗΣ]
Την προτιμούμε από την \texttt{ΟΣΟ} όταν θέλουμε να γίνει \textbf{τουλάχιστον μία} επανάληψη.
\end{tcolorbox}

\subsection{Σύνταξη}

\begin{center}
\begin{tabular}{|l|}
\hline
\texttt{ΑΡΧΗ\_ΕΠΑΝΑΛΗΨΗΣ} \\
\texttt{~~~~εντολές} \\
\texttt{ΜΕΧΡΙΣ\_ΟΤΟΥ συνθήκη} \\
\hline
\end{tabular}
\end{center}

\subsection{Παράδειγμα}

\begin{lstlisting}[language=,morekeywords={ΠΡΟΓΡΑΜΜΑ,ΜΕΤΑΒΛΗΤΕΣ,ΑΚΕΡΑΙΕΣ,ΑΡΧΗ,ΑΡΧΗ_ΕΠΑΝΑΛΗΨΗΣ,ΓΡΑΨΕ,ΔΙΑΒΑΣΕ,ΜΕΧΡΙΣ_ΟΤΟΥ,ΚΑΙ,ΤΕΛΟΣ_ΠΡΟΓΡΑΜΜΑΤΟΣ}]
ΠΡΟΓΡΑΜΜΑ ΗΕντολήΑΡΧΗ_ΕΠΑΝΑΛΗΨΗΣ 
!Διαβάζει έναν ακέραιο υποχρεώνοντας το 
!χρήστη να είναι μεταξύ 0 και 20. 
ΜΕΤΑΒΛΗΤΕΣ 
  ΑΚΕΡΑΙΕΣ: βαθμός 
ΑΡΧΗ 
  ΑΡΧΗ_ΕΠΑΝΑΛΗΨΗΣ 
    ΓΡΑΨΕ 'Δώσε βαθμό:  ' 
    ΔΙΑΒΑΣΕ βαθμός 
  ΜΕΧΡΙΣ_ΟΤΟΥ βαθμός >= 0 ΚΑΙ βαθμός <= 20 
  ΓΡΑΨΕ 'Έδωσες ', βαθμός 
ΤΕΛΟΣ_ΠΡΟΓΡΑΜΜΑΤΟΣ
\end{lstlisting}

\subsection{Παρατηρήσεις}

\begin{enumerate}[label=\arabic*.]
    \item \textbf{Έλεγχος συνθήκης:} Η συνθήκη ελέγχεται \textbf{μετά} από κάθε επανάληψη. Έτσι, οι εντολές μέσα στο βρόχο εκτελούνται \textbf{τουλάχιστον μία φορά}.
    
    \item \textbf{Συνθήκη τερματισμού:} Η επανάληψη σταματά όταν η συνθήκη γίνει \texttt{ΑΛΗΘΗΣ} (αντίθετα από την \texttt{ΟΣΟ} που συνεχίζει όσο είναι \texttt{ΑΛΗΘΗΣ}).
    
    \begin{tcolorbox}[colback=yellow!10,colframe=orange!50!black,title=Διαφορά από ΟΣΟ]
    \begin{itemize}
        \item \texttt{ΟΣΟ}: Επαναλαμβάνει \textbf{όσο} η συνθήκη είναι \texttt{ΑΛΗΘΗΣ}
        \item \texttt{ΑΡΧΗ\_ΕΠΑΝΑΛΗΨΗΣ}: Επαναλαμβάνει \textbf{μέχρι} η συνθήκη να γίνει \texttt{ΑΛΗΘΗΣ}
    \end{itemize}
    \end{tcolorbox}
    
    \item \textbf{Χωρίς ενσωματωμένο μετρητή:} Η \texttt{ΑΡΧΗ\_ΕΠΑΝΑΛΗΨΗΣ} δεν περιλαμβάνει μετρητή όπως η \texttt{ΓΙΑ}.
\end{enumerate}

\subsection{Χρήση μετρητή στην ΑΡΧΗ\_ΕΠΑΝΑΛΗΨΗΣ}

Αν θέλετε να βάλετε μετρητή στην \texttt{ΑΡΧΗ\_ΕΠΑΝΑΛΗΨΗΣ}, θυμηθείτε τα \textbf{τρία βήματα} του μετρητή:

\begin{center}
\begin{tabular}{|c|l|l|}
\hline
\textbf{Βήμα} & \textbf{Περιγραφή} & \textbf{Παράδειγμα} \\
\hline
1 & Αρχικοποίηση \textbf{πριν} την \texttt{ΑΡΧΗ\_ΕΠΑΝΑΛΗΨΗΣ} & \texttt{ι <- 1} \\
\hline
2 & Μεταβολή \textbf{πριν} το \texttt{ΜΕΧΡΙΣ\_ΟΤΟΥ} & \texttt{ι <- ι + 1} \\
\hline
3 & Συνθήκη τέλους \textbf{στη} \texttt{ΜΕΧΡΙΣ\_ΟΤΟΥ} & \texttt{ΜΕΧΡΙΣ\_ΟΤΟΥ ι = 10} \\
\hline
\end{tabular}
\end{center}

\begin{lstlisting}[language=,morekeywords={ΑΡΧΗ_ΕΠΑΝΑΛΗΨΗΣ,ΜΕΧΡΙΣ_ΟΤΟΥ},numbers=none]
ι <- 1                        ! 1. Αρχικοποίηση
ΑΡΧΗ_ΕΠΑΝΑΛΗΨΗΣ
  ! εντολές...
  ι <- ι + 1                  ! 2. Μεταβολή
ΜΕΧΡΙΣ_ΟΤΟΥ ι = 10            ! 3. Συνθήκη τερματισμού
\end{lstlisting}

\subsection{Ισοδυναμία με ΟΣΟ}

Η \texttt{ΑΡΧΗ\_ΕΠΑΝΑΛΗΨΗΣ ... ΜΕΧΡΙΣ\_ΟΤΟΥ} μπορεί να γραφεί ισοδύναμα με \texttt{ΟΣΟ}:

\begin{lstlisting}[language=,morekeywords={ΟΣΟ,ΕΠΑΝΑΛΑΒΕ,ΤΕΛΟΣ_ΕΠΑΝΑΛΗΨΗΣ,ΟΧΙ},numbers=none]
! Εκτέλεση μία φορά πριν τον έλεγχο
εντολές
ΟΣΟ ΟΧΙ συνθήκη ΕΠΑΝΑΛΑΒΕ
  εντολές
ΤΕΛΟΣ_ΕΠΑΝΑΛΗΨΗΣ
\end{lstlisting}

\begin{tcolorbox}[colback=backcolour,colframe=black,title=Σημείωση]
Παρατηρήστε ότι η συνθήκη αντιστρέφεται με το \texttt{ΟΧΙ}, γιατί η \texttt{ΟΣΟ} συνεχίζει όσο είναι \texttt{ΑΛΗΘΗΣ}, ενώ η \texttt{ΜΕΧΡΙΣ\_ΟΤΟΥ} σταματά όταν γίνει \texttt{ΑΛΗΘΗΣ}.
\end{tcolorbox}

% =======================================================
\section{Ενσωματωμένες συναρτήσεις}
% =======================================================

Η ΓΛΩΣΣΑ έχει ενσωματωμένες τις εξής συναρτήσεις:

\subsection{Λίστα συναρτήσεων}

\begin{itemize}
    \item \textbf{\texttt{Α\_Μ(πραγματικός)}}: το \textbf{Ακέραιο Μέρος} ενός πραγματικού. Δέχεται πραγματικό και επιστρέφει ακέραιο.
    
    \item \textbf{\texttt{Α\_Τ(αριθμός)}}: η \textbf{Απόλυτη Τιμή} ενός αριθμού. Δέχεται ακέραιο και επιστρέφει ακέραιο, ή δέχεται πραγματικό και επιστρέφει πραγματικό.
    
    \item \textbf{\texttt{Ε(πραγματικός)}}: η \textbf{Εκθετική} συνάρτηση ($e^x$). Επιστρέφει πραγματικό.
    
    \item \textbf{\texttt{ΕΦ(πραγματικός)}}: η \textbf{Εφαπτομένη} ενός πραγματικού. Επιστρέφει πραγματικό.
    
    \item \textbf{\texttt{ΗΜ(πραγματικός)}}: το \textbf{Ημίτονο} ενός πραγματικού. Επιστρέφει πραγματικό.
    
    \item \textbf{\texttt{ΛΟΓ(πραγματικός)}}: ο φυσικός \textbf{Λογάριθμος} ενός πραγματικού. Επιστρέφει πραγματικό.
    
    \item \textbf{\texttt{ΣΥΝ(πραγματικός)}}: το \textbf{Συνημίτονο} ενός πραγματικού. Επιστρέφει πραγματικό.
    
    \item \textbf{\texttt{Τ\_Ρ(πραγματικός)}}: η \textbf{Τετραγωνική Ρίζα} ενός πραγματικού. Ισοδυναμεί με $x^{1/2}$.
\end{itemize}

\subsection{Συνοπτικός πίνακας}

\begin{center}
\begin{tabular}{|l|l|l|l|}
\hline
\textbf{Συνάρτηση} & \textbf{Περιγραφή} & \textbf{Είσοδος} & \textbf{Έξοδος} \\
\hline
\texttt{Α\_Μ(x)} & Ακέραιο μέρος & Πραγματικός & Ακέραιος \\
\hline
\texttt{Α\_Τ(x)} & Απόλυτη τιμή & Ακέρ./Πραγμ. & Ίδιος τύπος \\
\hline
\texttt{Ε(x)} & Εκθετική ($e^x$) & Πραγματικός & Πραγματικός \\
\hline
\texttt{ΕΦ(x)} & Εφαπτομένη & Πραγματικός (μοίρες) & Πραγματικός \\
\hline
\texttt{ΗΜ(x)} & Ημίτονο & Πραγματικός (μοίρες) & Πραγματικός \\
\hline
\texttt{ΛΟΓ(x)} & Φυσικός λογάριθμος & Πραγματικός ($>0$) & Πραγματικός \\
\hline
\texttt{ΣΥΝ(x)} & Συνημίτονο & Πραγματικός (μοίρες) & Πραγματικός \\
\hline
\texttt{Τ\_Ρ(x)} & Τετραγωνική ρίζα & Πραγματικός ($\geq 0$) & Πραγματικός \\
\hline
\end{tabular}
\end{center}

\subsection{Παράδειγμα}

\begin{lstlisting}[language=,morekeywords={ΠΡΟΓΡΑΜΜΑ,ΜΕΤΑΒΛΗΤΕΣ,ΠΡΑΓΜΑΤΙΚΕΣ,ΑΡΧΗ,ΓΡΑΨΕ,ΤΕΛΟΣ_ΠΡΟΓΡΑΜΜΑΤΟΣ}]
ΠΡΟΓΡΑΜΜΑ ΕνσωματωμένεςΣυναρτήσεις 
!Επιδεικνύει τις ενσωματωμένες συναρτήσεις της Γλώσσας 
ΜΕΤΑΒΛΗΤΕΣ 
  ΠΡΑΓΜΑΤΙΚΕΣ: π 
ΑΡΧΗ 
  π <- -3.2 
  ΓΡΑΨΕ 'Ακέραιο μέρος του ', π, ': ', Α_Μ(π) 
  ΓΡΑΨΕ 'Απόλυτη τιμή του ', π, ': ', Α_Τ(π) 
  ΓΡΑΨΕ 'Εκθετική συνάρτηση του ', π, ': ', Ε(π) 
  ΓΡΑΨΕ 'Εφαπτομένη του ', π, ': ', ΕΦ(π) 
  ΓΡΑΨΕ 'Ημίτονο του ', π, ': ', ΗΜ(π) 
!Στο δεκαδικό λογάριθμο δεν επιτρέπονται αρνητικοί αριθμοί 
  ΓΡΑΨΕ 'Δεκαδικός λογάριθμος του ', Α_Τ(π), ': ', ΛΟΓ(Α_Τ(π)) 
  ΓΡΑΨΕ 'Συνημίτονο του ', π, ': ', ΣΥΝ(π) 
!Στην τετραγωνική ρίζα δεν επιτρέπονται αρνητικοί αριθμοί 
  ΓΡΑΨΕ 'Τετραγωνική ρίζα του ', Α_Τ(π), ': ', Τ_Ρ(Α_Τ(π)) 
ΤΕΛΟΣ_ΠΡΟΓΡΑΜΜΑΤΟΣ
\end{lstlisting}

\subsection{Παρατηρήσεις}

\begin{enumerate}[label=\arabic*.]
    \item \textbf{Ακέραιοι ως πραγματικοί:} Όπου μπορεί να μπει πραγματικός, μπορεί και ακέραιος, λόγω του ότι οι ακέραιοι είναι υποσύνολο των πραγματικών. Το ίδιο ισχύει και στην ανάθεση τιμής \texttt{π <- 1} όπου το \texttt{π} είναι πραγματικός και το \texttt{1} ακέραιος.
    
    \item \textbf{Τριγωνομετρικές σε μοίρες:} Η παράμετρος των \texttt{ΕΦ}, \texttt{ΗΜ} και \texttt{ΣΥΝ} είναι σε \textbf{μοίρες}, όχι σε ακτίνια (σελίδα 62 του τετραδίου μαθητή).
    
    \begin{tcolorbox}[colback=yellow!10,colframe=orange!50!black,title=Σημείωση]
    Μπορείτε να αλλάξετε αυτή τη συμπεριφορά από τις επιλογές του Διερμηνευτή, αλλά \textbf{δεν προτείνεται}.
    \end{tcolorbox}
    
    \item \textbf{Ασάφεια στην Α\_Μ:} Υπάρχει ασάφεια σχετικά με το αν η \texttt{Α\_Μ(-5.5)} επιστρέφει \texttt{-5} ή \texttt{-6}. Μπορείτε να ορίσετε τη συμπεριφορά που θέλετε από τις επιλογές του Διερμηνευτή.
    
    \item \textbf{Περιορισμοί τιμών:}
    \begin{itemize}
        \item Στη \texttt{ΛΟΓ} δεν επιτρέπονται αρνητικοί αριθμοί ή μηδέν
        \item Στην \texttt{Τ\_Ρ} δεν επιτρέπονται αρνητικοί αριθμοί
    \end{itemize}
    
    \item \textbf{Σταθερή αποτίμηση:} Οι ενσωματωμένες συναρτήσεις είναι \textbf{σταθερής αποτίμησης}, δηλαδή μπορούν να χρησιμοποιηθούν σε:
    \begin{itemize}
        \item Δηλώσεις σταθερών
        \item Όρια πινάκων
        \item Εντολές \texttt{ΠΕΡΙΠΤΩΣΗ}
    \end{itemize}
    
    \begin{lstlisting}[language=,morekeywords={ΣΤΑΘΕΡΕΣ,ΜΕΤΑΒΛΗΤΕΣ,ΑΚΕΡΑΙΕΣ},numbers=none]
ΣΤΑΘΕΡΕΣ
  Ν = 10
  ρίζα_Ν = Τ_Ρ(Ν)
ΜΕΤΑΒΛΗΤΕΣ
  ΑΚΕΡΑΙΕΣ: Α[Α_Μ(Τ_Ρ(100))]    ! Πίνακας 10 θέσεων
    \end{lstlisting}
\end{enumerate}

% =======================================================
\section{Η εντολή ΚΑΛΕΣΕ}
% =======================================================

Η εντολή \texttt{ΚΑΛΕΣΕ} μας επιτρέπει να καλέσουμε μία \textbf{διαδικασία}.

\begin{tcolorbox}[colback=yellow!10,colframe=orange!50!black,title=Σημείωση]
Η \texttt{ΚΑΛΕΣΕ} χρησιμοποιείται \textbf{μόνο} για διαδικασίες, όχι για συναρτήσεις. Οι συναρτήσεις καλούνται μέσα από εκφράσεις.
\end{tcolorbox}

\subsection{Σύνταξη}

\begin{center}
\begin{tabular}{|l|}
\hline
\texttt{ΚΑΛΕΣΕ ΌνομαΔιαδικασίας(παράμετρος1, παράμετρος2, ...)} \\
\hline
\end{tabular}
\end{center}

\subsection{Παράδειγμα}

\begin{lstlisting}[language=,morekeywords={ΠΡΟΓΡΑΜΜΑ,ΑΡΧΗ,ΚΑΛΕΣΕ,ΤΕΛΟΣ_ΠΡΟΓΡΑΜΜΑΤΟΣ,ΔΙΑΔΙΚΑΣΙΑ,ΜΕΤΑΒΛΗΤΕΣ,ΧΑΡΑΚΤΗΡΕΣ,ΓΡΑΨΕ,ΤΕΛΟΣ_ΔΙΑΔΙΚΑΣΙΑΣ}]
ΠΡΟΓΡΑΜΜΑ ΗΕντολήΚάλεσε 
!Επιδεικνύει απλά τη χρήση της εντολής. 
ΑΡΧΗ 
  ΚΑΛΕΣΕ ΓράψεΌνομα('My name is Bond. James Bond.') 
ΤΕΛΟΣ_ΠΡΟΓΡΑΜΜΑΤΟΣ 

ΔΙΑΔΙΚΑΣΙΑ ΓράψεΌνομα(όνομα) 
ΜΕΤΑΒΛΗΤΕΣ 
  ΧΑΡΑΚΤΗΡΕΣ: όνομα 
ΑΡΧΗ 
  ΓΡΑΨΕ όνομα 
ΤΕΛΟΣ_ΔΙΑΔΙΚΑΣΙΑΣ
\end{lstlisting}

\subsection{Παρατηρήσεις}

\begin{enumerate}[label=\arabic*.]
    \item \textbf{Παράμετροι:} Ο αριθμός και ο τύπος των παραμέτρων καθορίζονται από την αντίστοιχη διαδικασία.
    
    \item \textbf{Διαδικασία χωρίς παραμέτρους:} Αν μία διαδικασία δεν έχει παραμέτρους, \textbf{δεν πρέπει} να μπουν παρενθέσεις κατά την κλήση της.
    
    \begin{lstlisting}[language=,morekeywords={ΚΑΛΕΣΕ},numbers=none]
ΚΑΛΕΣΕ ΔιαδικασίαΧωρίςΠαραμέτρους    ! Σωστό
ΚΑΛΕΣΕ ΔιαδικασίαΧωρίςΠαραμέτρους()  ! Λάθος!
    \end{lstlisting}
\end{enumerate}

\subsection{Σύγκριση διαδικασιών και συναρτήσεων}

\begin{center}
\begin{tabular}{|l|c|c|}
\hline
\textbf{Χαρακτηριστικό} & \textbf{Διαδικασία} & \textbf{Συνάρτηση} \\
\hline
Κλήση με \texttt{ΚΑΛΕΣΕ} & \checkmark & $\times$ \\
\hline
Κλήση μέσα σε έκφραση & $\times$ & \checkmark \\
\hline
Επιστρέφει τιμή & $\times$ & \checkmark \\
\hline
Πολλαπλές έξοδοι (με παραμέτρους) & \checkmark & $\times$ \\
\hline
\end{tabular}
\end{center}

% =======================================================
\section{Διαδικασίες}
% =======================================================

Οι διαδικασίες είναι \textbf{υποπρογράμματα} που γράφονται μετά το κυρίως πρόγραμμα. Μπορούν να κληθούν με την εντολή \texttt{ΚΑΛΕΣΕ} από οποιοδήποτε σημείο του προγράμματος.

\subsection{Σύνταξη}

\begin{center}
\begin{tabular}{|l|}
\hline
\texttt{ΔΙΑΔΙΚΑΣΙΑ ΌνομαΔιαδικασίας(παράμετρος1, παράμετρος2, ...)} \\
\texttt{ΣΤΑΘΕΡΕΣ} \\
\texttt{~~~~...} \\
\texttt{ΜΕΤΑΒΛΗΤΕΣ} \\
\texttt{~~~~τύπος: παράμετρος1, παράμετρος2, ...} \\
\texttt{~~~~τύπος: τοπικές\_μεταβλητές} \\
\texttt{ΑΡΧΗ} \\
\texttt{~~~~εντολές} \\
\texttt{ΤΕΛΟΣ\_ΔΙΑΔΙΚΑΣΙΑΣ} \\
\hline
\end{tabular}
\end{center}

\begin{tcolorbox}[colback=yellow!10,colframe=orange!50!black,title=Σημείωση]
Οι παράμετροι αναφέρονται ονομαστικά κατά τη δήλωση της διαδικασίας και στη συνέχεια ο τύπος τους δηλώνεται στο τμήμα \texttt{ΜΕΤΑΒΛΗΤΕΣ}.
\end{tcolorbox}

\subsection{Παράδειγμα}

\begin{lstlisting}[language=,morekeywords={ΠΡΟΓΡΑΜΜΑ,ΑΡΧΗ,ΚΑΛΕΣΕ,ΤΕΛΟΣ_ΠΡΟΓΡΑΜΜΑΤΟΣ,ΔΙΑΔΙΚΑΣΙΑ,ΜΕΤΑΒΛΗΤΕΣ,ΧΑΡΑΚΤΗΡΕΣ,ΓΡΑΨΕ,ΤΕΛΟΣ_ΔΙΑΔΙΚΑΣΙΑΣ}]
ΠΡΟΓΡΑΜΜΑ ΜίαΔιαδικασία 
!Δημιουργία διαδικασίας και κλήση της με την Κάλεσε. 
ΑΡΧΗ 
  ΚΑΛΕΣΕ ΓράψεΌνομα('My name is Bond. James Bond.') 
ΤΕΛΟΣ_ΠΡΟΓΡΑΜΜΑΤΟΣ 

ΔΙΑΔΙΚΑΣΙΑ ΓράψεΌνομα(όνομα) 
ΜΕΤΑΒΛΗΤΕΣ 
  ΧΑΡΑΚΤΗΡΕΣ: όνομα 
ΑΡΧΗ 
  ΓΡΑΨΕ όνομα 
ΤΕΛΟΣ_ΔΙΑΔΙΚΑΣΙΑΣ
\end{lstlisting}

\subsection{Παρατηρήσεις}

\begin{enumerate}[label=\arabic*.]
    \item \textbf{Τι μπορεί να περαστεί ως παράμετρος:} Σαν παράμετρος μπορεί να περαστεί οτιδήποτε έχει τιμή:
    \begin{itemize}
        \item Μεταβλητές
        \item Πίνακες
        \item Σταθερές
        \item Εκφράσεις
        \item Εκφράσεις που περιέχουν κλήση συναρτήσεων
    \end{itemize}
    
    \item \textbf{Μεταβίβαση με αναφορά:} Αν μία διαδικασία κληθεί με \textbf{μεταβλητή} σαν παράμετρο και αλλαχθεί η τιμή της μέσα στη διαδικασία, τότε η τιμή της μεταβλητής στο καλών υποπρόγραμμα θα \textbf{ενημερωθεί} μετά την εκτέλεση της εντολής \texttt{ΤΕΛΟΣ\_ΔΙΑΔΙΚΑΣΙΑΣ}.
    
    \begin{tcolorbox}[colback=blue!5,colframe=blue!50!black,title=Παράδειγμα μεταβίβασης]
\begin{lstlisting}[language=,morekeywords={ΠΡΟΓΡΑΜΜΑ,ΜΕΤΑΒΛΗΤΕΣ,ΑΚΕΡΑΙΕΣ,ΑΡΧΗ,ΚΑΛΕΣΕ,ΓΡΑΨΕ,ΤΕΛΟΣ_ΠΡΟΓΡΑΜΜΑΤΟΣ,ΔΙΑΔΙΚΑΣΙΑ,ΤΕΛΟΣ_ΔΙΑΔΙΚΑΣΙΑΣ},numbers=none,frame=none,backgroundcolor=\color{white}]
ΠΡΟΓΡΑΜΜΑ Παράδειγμα
ΜΕΤΑΒΛΗΤΕΣ
  ΑΚΕΡΑΙΕΣ: x
ΑΡΧΗ
  x <- 5
  ΚΑΛΕΣΕ Διπλασίασε(x)
  ΓΡΑΨΕ x              ! Εμφανίζει 10
ΤΕΛΟΣ_ΠΡΟΓΡΑΜΜΑΤΟΣ

ΔΙΑΔΙΚΑΣΙΑ Διπλασίασε(α)
ΜΕΤΑΒΛΗΤΕΣ
  ΑΚΕΡΑΙΕΣ: α
ΑΡΧΗ
  α <- α * 2
ΤΕΛΟΣ_ΔΙΑΔΙΚΑΣΙΑΣ
\end{lstlisting}
    \end{tcolorbox}
    
    \item \textbf{Διαδικασία χωρίς παραμέτρους:} Μία διαδικασία χωρίς παραμέτρους δηλώνεται και καλείται \textbf{χωρίς παρενθέσεις}.
    
\begin{lstlisting}[language=,morekeywords={ΔΙΑΔΙΚΑΣΙΑ,ΑΡΧΗ,ΤΕΛΟΣ_ΔΙΑΔΙΚΑΣΙΑΣ,ΚΑΛΕΣΕ},numbers=none]
ΔΙΑΔΙΚΑΣΙΑ Χαιρετισμός    ! Χωρίς παρενθέσεις
ΑΡΧΗ
  ΓΡΑΨΕ 'Γεια σου!'
ΤΕΛΟΣ_ΔΙΑΔΙΚΑΣΙΑΣ

! Κλήση:
ΚΑΛΕΣΕ Χαιρετισμός        ! Χωρίς παρενθέσεις
\end{lstlisting}
    
    \item \textbf{Αναδρομή:} Επιτρέπονται μέχρι \textbf{1000 αναδρομικές κλήσεις} (μία διαδικασία να καλεί τον εαυτό της). Αν γίνουν παραπάνω, εμφανίζεται το μήνυμα «Υπερχείλιση στοίβας» και το πρόγραμμα τερματίζεται.
    
    \begin{tcolorbox}[colback=backcolour,colframe=black,title=Σημείωση]
    Οι \textbf{επαναληπτικές κλήσεις} (μία διαδικασία καλεί άλλη διαδικασία που καλεί την πρώτη κ.ο.κ.) επιτρέπονται απεριόριστες.
    \end{tcolorbox}
\end{enumerate}

% =======================================================
\section{Συναρτήσεις}
% =======================================================

Η ΓΛΩΣΣΑ επιτρέπει τη δημιουργία \textbf{συναρτήσεων} από το χρήστη, οι οποίες μπορούν στη συνέχεια να χρησιμοποιηθούν όπως και οι ενσωματωμένες συναρτήσεις.

Το αποτέλεσμα κάθε συνάρτησης έχει πάντα τιμή κάποιου τύπου δεδομένων και δηλώνεται μετά τις παραμέτρους της συνάρτησης (\texttt{ΑΚΕΡΑΙΑ}, \texttt{ΠΡΑΓΜΑΤΙΚΗ}, \texttt{ΧΑΡΑΚΤΗΡΑΣ} ή \texttt{ΛΟΓΙΚΗ}).

\begin{tcolorbox}[colback=red!10,colframe=red!50!black,title=Προσοχή]
Οι συναρτήσεις \textbf{δεν επιτρέπεται} να επιστρέφουν πίνακες.
\end{tcolorbox}

\subsection{Σύνταξη}

\begin{center}
\begin{tabular}{|l|}
\hline
\texttt{ΣΥΝΑΡΤΗΣΗ ΌνομαΣυνάρτησης(παράμ1, παράμ2, ...): ΤΥΠΟΣ} \\
\texttt{ΣΤΑΘΕΡΕΣ} \\
\texttt{~~~~...} \\
\texttt{ΜΕΤΑΒΛΗΤΕΣ} \\
\texttt{~~~~τύπος: παράμ1, παράμ2, ...} \\
\texttt{~~~~τύπος: τοπικές\_μεταβλητές} \\
\texttt{ΑΡΧΗ} \\
\texttt{~~~~εντολές} \\
\texttt{~~~~ΌνομαΣυνάρτησης <- τιμή} \\
\texttt{ΤΕΛΟΣ\_ΣΥΝΑΡΤΗΣΗΣ} \\
\hline
\end{tabular}
\end{center}

\begin{tcolorbox}[colback=yellow!10,colframe=orange!50!black,title=Υποχρεωτική επιστροφή τιμής]
Όλες οι συναρτήσεις \textbf{πρέπει} να περιέχουν μία εντολή του τύπου \texttt{ΌνομαΣυνάρτησης <- τιμή}, ώστε να επιστρέφουν κάποια τιμή στο καλών υποπρόγραμμα.
\end{tcolorbox}

\subsection{Παράδειγμα}

\begin{lstlisting}[language=,morekeywords={ΠΡΟΓΡΑΜΜΑ,ΑΡΧΗ,ΓΡΑΨΕ,ΤΕΛΟΣ_ΠΡΟΓΡΑΜΜΑΤΟΣ,ΣΥΝΑΡΤΗΣΗ,ΑΚΕΡΑΙΑ,ΜΕΤΑΒΛΗΤΕΣ,ΑΚΕΡΑΙΕΣ,ΑΝ,ΤΟΤΕ,ΑΛΛΙΩΣ,ΤΕΛΟΣ_ΑΝ,ΤΕΛΟΣ_ΣΥΝΑΡΤΗΣΗΣ}]
ΠΡΟΓΡΑΜΜΑ Συναρτήσεις 
!Παράδειγμα υλοποίησης συναρτήσεων. 
!Υλοποιεί τις συναρτήσεις ΑπόλυτηΤιμή και Μέγιστος. 
ΑΡΧΗ 
  ΓΡΑΨΕ Μέγιστος(ΑπόλυτηΤιμή(-2), 1)              !Θα γράψει «2» 
ΤΕΛΟΣ_ΠΡΟΓΡΑΜΜΑΤΟΣ 

ΣΥΝΑΡΤΗΣΗ ΑπόλυτηΤιμή(χ): ΑΚΕΡΑΙΑ 
!Βρίσκει την απόλυτη τιμή του χ χωρίς να χρησιμοποιήσει την Α_Τ 
ΜΕΤΑΒΛΗΤΕΣ 
  ΑΚΕΡΑΙΕΣ: χ 
ΑΡΧΗ 
  ΑΝ χ >= 0 ΤΟΤΕ 
    ΑπόλυτηΤιμή <- χ 
  ΑΛΛΙΩΣ 
    ΑπόλυτηΤιμή <- -χ 
  ΤΕΛΟΣ_ΑΝ 
ΤΕΛΟΣ_ΣΥΝΑΡΤΗΣΗΣ 

ΣΥΝΑΡΤΗΣΗ Μέγιστος(α, β): ΑΚΕΡΑΙΑ 
!Επιστρέφει το Μέγιστο μεταξύ των α και β 
ΜΕΤΑΒΛΗΤΕΣ 
  ΑΚΕΡΑΙΕΣ: α, β 
ΑΡΧΗ 
  ΑΝ α >= β ΤΟΤΕ 
    Μέγιστος <- α 
  ΑΛΛΙΩΣ 
    Μέγιστος <- β 
  ΤΕΛΟΣ_ΑΝ 
ΤΕΛΟΣ_ΣΥΝΑΡΤΗΣΗΣ
\end{lstlisting}

\subsection{Παρατηρήσεις}

\begin{enumerate}[label=\arabic*.]
    \item \textbf{Υποχρεωτική επιστροφή:} Αν δεν υπάρχει εντολή τύπου \texttt{ΌνομαΣυνάρτησης <- τιμή} στο εσωτερικό μιας συνάρτησης, ο Διερμηνευτής εμφανίζει \textbf{σφάλμα εκτέλεσης}.
    
    \item \textbf{Αναδρομή με το όνομα:} Αν το όνομα μιας συνάρτησης εμφανιστεί οπουδήποτε αλλού εκτός από το \textbf{αριστερό μέρος} μιας ανάθεσης τιμής, προκαλείται αναδρομή.
    
    \begin{lstlisting}[language=,morekeywords={ΣΥΝΑΡΤΗΣΗ,ΑΚΕΡΑΙΑ,ΑΡΧΗ,ΑΝ,ΤΟΤΕ,ΑΛΛΙΩΣ,ΤΕΛΟΣ_ΑΝ,ΤΕΛΟΣ_ΣΥΝΑΡΤΗΣΗΣ},numbers=none]
ΣΥΝΑΡΤΗΣΗ Παραγοντικό(ν): ΑΚΕΡΑΙΑ
ΑΡΧΗ
  ΑΝ ν <= 1 ΤΟΤΕ
    Παραγοντικό <- 1           ! Επιστροφή τιμής
  ΑΛΛΙΩΣ
    Παραγοντικό <- ν * Παραγοντικό(ν-1)  ! Αναδρομή!
  ΤΕΛΟΣ_ΑΝ
ΤΕΛΟΣ_ΣΥΝΑΡΤΗΣΗΣ
    \end{lstlisting}
    
    \item \textbf{Τι μπορεί να περαστεί ως παράμετρος:} Σαν παράμετρος μπορεί να περαστεί οτιδήποτε έχει τιμή:
    \begin{itemize}
        \item Μεταβλητές
        \item Πίνακες
        \item Σταθερές
        \item Εκφράσεις
        \item Εκφράσεις που περιέχουν κλήση συναρτήσεων
    \end{itemize}
    
    \item \textbf{Μεταβίβαση με τιμή:} Οι παράμετροι στις συναρτήσεις περνιούνται \textbf{με τιμή}, δηλαδή οποιαδήποτε αλλαγή στις τυπικές παραμέτρους μιας συνάρτησης \textbf{δεν επηρεάζει} τις πραγματικές παραμέτρους.
    
    \begin{tcolorbox}[colback=blue!5,colframe=blue!50!black,title=Διαφορά από διαδικασίες]
    Στις \textbf{διαδικασίες} οι παράμετροι περνιούνται με αναφορά (οι αλλαγές επηρεάζουν το καλών πρόγραμμα), ενώ στις \textbf{συναρτήσεις} με τιμή (οι αλλαγές δεν επηρεάζουν).
    \end{tcolorbox}
    
    \item \textbf{Συνάρτηση χωρίς παραμέτρους:} Μία συνάρτηση χωρίς παραμέτρους δηλώνεται και καλείται \textbf{χωρίς παρενθέσεις}.
    
    \item \textbf{Αναδρομή:} Επιτρέπονται μέχρι \textbf{1000 αναδρομικές κλήσεις}. Αν γίνουν παραπάνω, εμφανίζεται το μήνυμα «Υπερχείλιση στοίβας» και το πρόγραμμα τερματίζεται.
\end{enumerate}

\subsection{Σύγκριση διαδικασιών και συναρτήσεων}

\begin{center}
\begin{tabular}{|l|c|c|}
\hline
\textbf{Χαρακτηριστικό} & \textbf{Διαδικασία} & \textbf{Συνάρτηση} \\
\hline
Κλήση & με \texttt{ΚΑΛΕΣΕ} & μέσα σε έκφραση \\
\hline
Επιστρέφει τιμή & $\times$ & \checkmark \\
\hline
Μπορεί να επιστρέψει πίνακα & $\times$ & $\times$ \\
\hline
Μεταβίβαση παραμέτρων & με αναφορά & με τιμή \\
\hline
Αλλαγές επηρεάζουν καλών & \checkmark & $\times$ \\
\hline
Χωρίς παραμέτρους & χωρίς () & χωρίς () \\
\hline
Όριο αναδρομής & 1000 & 1000 \\
\hline
\end{tabular}
\end{center}

% =======================================================
\section{Μεταβίβαση παραμέτρων σε υποπρογράμματα}
% =======================================================

Η ΓΛΩΣΣΑ μοιάζει λίγο με την Basic στη μεταβίβαση παραμέτρων, αλλά είναι πάρα πολύ διαφορετική από την Pascal. Η Basic ακολουθεί το μηχανισμό μεταβίβασης παραμέτρων \textbf{με αναφορά}, όποτε αυτό είναι δυνατό.

\subsection{Παράδειγμα}

\begin{lstlisting}[language=,morekeywords={ΠΡΟΓΡΑΜΜΑ,ΜΕΤΑΒΛΗΤΕΣ,ΠΡΑΓΜΑΤΙΚΕΣ,ΑΚΕΡΑΙΕΣ,ΑΡΧΗ,ΓΡΑΨΕ,ΚΑΛΕΣΕ,ΤΕΛΟΣ_ΠΡΟΓΡΑΜΜΑΤΟΣ,ΔΙΑΔΙΚΑΣΙΑ,ΤΕΛΟΣ_ΔΙΑΔΙΚΑΣΙΑΣ}]
ΠΡΟΓΡΑΜΜΑ ΜεταβίβασηΠαραμέτρων 
ΜΕΤΑΒΛΗΤΕΣ 
  ΠΡΑΓΜΑΤΙΚΕΣ: π 
  ΑΚΕΡΑΙΕΣ: α 
ΑΡΧΗ 
  π <- 1 
  ΓΡΑΨΕ 'π = ', π                                      !Γράφει 1 
  ΚΑΛΕΣΕ Δ(π)         !Καλείται με μεταβλητή, οπότε το π αλλάζει 
  ΓΡΑΨΕ 'π = ', π                                      !Γράφει 2 
  α <- 1 
  ΓΡΑΨΕ 'α = ', α                                      !Γράφει 1 
! Κάλεσε Δ(α)    !Αυτό δεν επιτρέπεται, προσπαθούμε να περάσουμε 
                 !με αναφορά ακέραιο ενώ απαιτείται πραγματικός. 
  ΚΑΛΕΣΕ Δ((α))  !Βάζοντας το α σε παρένθεση το κάνουμε έκφραση, 
!οπότε περνιέται με τιμή. Το ίδιο θα γινόταν με Κάλεσε Δ(α + 0). 
  ΓΡΑΨΕ 'α = ', α 
ΤΕΛΟΣ_ΠΡΟΓΡΑΜΜΑΤΟΣ 

ΔΙΑΔΙΚΑΣΙΑ Δ(β) 
!Αυξάνει το β. Αν περαστεί μεταβλητή (= με αναφορά) αλλάζει και 
!η μεταβλητή του κυρίως προγράμματος. Αν περαστεί έκφραση (= με 
!τιμή) το β αλλάζει μόνο τοπικά. 
ΜΕΤΑΒΛΗΤΕΣ 
  ΠΡΑΓΜΑΤΙΚΕΣ: β 
ΑΡΧΗ 
  β <- β + 1 
ΤΕΛΟΣ_ΔΙΑΔΙΚΑΣΙΑΣ Δ
\end{lstlisting}

\subsection{Κανόνες μεταβίβασης}

\subsubsection{Μεταβίβαση με αναφορά (μεταβλητή ως παράμετρος)}

Αν καλέσουμε με παράμετρο \textbf{μεταβλητή} κάποια \textbf{διαδικασία} (όχι συνάρτηση) και αυτή αλλάξει την τιμή της, η νέα τιμή θα επιστραφεί στο κυρίως πρόγραμμα.

\begin{tcolorbox}[colback=red!10,colframe=red!50!black,title=Υποχρεωτική συμβατότητα τύπων]
Η τυπική και η ουσιαστική παράμετρος \textbf{πρέπει} να είναι του \textbf{ίδιου τύπου} δεδομένων. Δεν επιτρέπεται καν να περάσουμε ακέραιο σε διαδικασία που περιμένει πραγματικό αριθμό!
\end{tcolorbox}

\subsubsection{Μεταβίβαση με τιμή (έκφραση ως παράμετρος)}

Αν καλέσουμε με παράμετρο \textbf{έκφραση} κάποιο υποπρόγραμμα (είτε διαδικασία είτε συνάρτηση) και αυτό αλλάξει την τιμή της, η νέα τιμή \textbf{δεν μπορεί} να επιστραφεί στο κυρίως πρόγραμμα.

\begin{tcolorbox}[colback=yellow!10,colframe=orange!50!black,title=Ευελιξία τύπων στις εκφράσεις]
Η έκφραση πρέπει να είναι του ίδιου τύπου με την τυπική παράμετρο, με τη διαφορά ότι \textbf{επιτρέπεται} πέρασμα ακεραίου σε υποπρόγραμμα που περιμένει πραγματικό (όπως και στην ανάθεση τιμής).
\end{tcolorbox}

\subsubsection{Μετατροπή μεταβλητής σε έκφραση}

Αν θέλουμε να περάσουμε με τιμή κάποια μεταβλητή, πρέπει να κάνουμε κάποια «πράξη» ώστε να πάψει να είναι μεταβλητή:

\begin{center}
\begin{tabular}{|l|l|}
\hline
\textbf{Τρόπος} & \textbf{Παράδειγμα} \\
\hline
Παρένθεση (απλούστερο) & \texttt{ΚΑΛΕΣΕ Δ((χ))} \\
\hline
Πρόσθεση μηδέν & \texttt{ΚΑΛΕΣΕ Δ(χ + 0)} \\
\hline
Πολλαπλασιασμός με ένα & \texttt{ΚΑΛΕΣΕ Δ(1 * χ)} \\
\hline
\end{tabular}
\end{center}

\subsection{Συνοπτικός πίνακας}

\begin{center}
\begin{tabular}{|l|c|c|}
\hline
\textbf{Χαρακτηριστικό} & \textbf{Μεταβλητή} & \textbf{Έκφραση} \\
\hline
Μηχανισμός & Με αναφορά & Με τιμή \\
\hline
Αλλαγές επηρεάζουν καλών & \checkmark (διαδικασίες) & $\times$ \\
\hline
Ακέραιος $\rightarrow$ Πραγματικός & $\times$ & \checkmark \\
\hline
\end{tabular}
\end{center}

\subsection{Ο μηχανισμός copy in - copy out}

Το σχολικό βιβλίο (σελ. 218) περιγράφει μηχανισμό μεταβίβασης παραμέτρων με \textbf{αντιγραφή} (copy in - copy out), όχι με αναφορά. Αυτό ακριβώς έχει υλοποιηθεί στο Διερμηνευτή.

\subsubsection{Διαφορές copy in - copy out από αναφορά}

\begin{enumerate}[label=\arabic*.]
    \item \textbf{Χρονική στιγμή ενημέρωσης:}
    
    \begin{itemize}
        \item \textbf{Με αναφορά:} Όταν αλλάζει η τιμή μιας παραμέτρου, \textbf{ταυτόχρονα} γίνεται η αλλαγή στην αντίστοιχη μεταβλητή του κυρίως προγράμματος (δεν υπάρχουν δύο μεταβλητές, απλά δύο ονόματα για την ίδια).
        
        \item \textbf{Copy in - copy out:} Υπάρχουν \textbf{δύο μεταβλητές} και η δεύτερη (της διαδικασίας) αντιγράφεται στην πρώτη \textbf{μόνο} με την εντολή \texttt{ΤΕΛΟΣ\_ΔΙΑΔΙΚΑΣΙΑΣ}.
    \end{itemize}
    
    \begin{tcolorbox}[colback=blue!5,colframe=blue!50!black,title=Πείραμα στο Διερμηνευτή]
    Σταματήστε την εκτέλεση αφού ξεκινήσει η \texttt{ΚΑΛΕΣΕ Δ(π)} αλλά πριν το \texttt{ΤΕΛΟΣ\_ΔΙΑΔΙΚΑΣΙΑΣ}. Το \texttt{β} θα έχει γίνει 2, αλλά το \texttt{π} θα έχει ακόμα την τιμή 1. Γίνεται 2 \textbf{μετά} το \texttt{ΤΕΛΟΣ\_ΔΙΑΔΙΚΑΣΙΑΣ}.
    \end{tcolorbox}
    
    \item \textbf{Κατανάλωση μνήμης:}
    
    Η δημιουργία διαφορετικών μεταβλητών για κάθε κλήση σπαταλά μνήμη. Π.χ. ταξινόμηση πίνακα 5000 ονομάτων με QuickSort (1 Mb) και 20 αναδρομικές κλήσεις θα χρειαστεί 20 Mb RAM μόνο για τον πίνακα.
    
    \item \textbf{Διπλή παράμετρος με ίδια μεταβλητή:}
    
    \begin{tcolorbox}[colback=red!10,colframe=red!50!black,title=Κρίσιμη διαφορά]
    Αν έχουμε διαδικασία που αυξάνει δύο παραμέτρους (\texttt{α <- α + 1}, \texttt{β <- β + 1}) και την καλέσουμε με την ίδια μεταβλητή (\texttt{ΚΑΛΕΣΕ Αύξηση(α, α)}):
    \begin{itemize}
        \item \textbf{Με αναφορά:} το \texttt{α} θα αυξηθεί κατά \textbf{2}
        \item \textbf{Copy in - copy out:} το \texttt{α} θα αυξηθεί κατά \textbf{1}
    \end{itemize}
    Αυτό συμβαίνει επειδή δημιουργούνται δύο αντίγραφα του \texttt{α}, αυξάνονται και τα δύο κατά ένα, και τελικά το ένα αντικαθιστά το άλλο.
    \end{tcolorbox}
\end{enumerate}

\begin{tcolorbox}[colback=yellow!10,colframe=orange!50!black,title=Σύσταση]
Καλύτερα να αποφεύγεται η κατασκευή ασκήσεων που χρησιμοποιούν αυτήν τη δυσνόητη συμπεριφορά.
\end{tcolorbox}

% =======================================================
\section{Λέξεις-κλειδιά της ΓΛΩΣΣΑΣ}
% =======================================================

Σε αυτήν την ενότητα παρουσιάζονται όλες οι \textbf{λέξεις-κλειδιά} (δεσμευμένες λέξεις) της ΓΛΩΣΣΑΣ, οργανωμένες σε κατηγορίες για εύκολη αναφορά.

% -------------------------------------------------------
\subsection{Δομή προγράμματος}
% -------------------------------------------------------

\begin{center}
\begin{tabular}{|l|l|}
\hline
\textbf{Λέξη-κλειδί} & \textbf{Περιγραφή} \\
\hline
\texttt{ΠΡΟΓΡΑΜΜΑ} & Έναρξη προγράμματος (επικεφαλίδα) \\
\hline
\texttt{ΣΤΑΘΕΡΕΣ} & Τμήμα δήλωσης σταθερών \\
\hline
\texttt{ΜΕΤΑΒΛΗΤΕΣ} & Τμήμα δήλωσης μεταβλητών \\
\hline
\texttt{ΑΡΧΗ} & Έναρξη κυρίως σώματος \\
\hline
\texttt{ΤΕΛΟΣ\_ΠΡΟΓΡΑΜΜΑΤΟΣ} & Τέλος προγράμματος \\
\hline
\end{tabular}
\end{center}

% -------------------------------------------------------
\subsection{Τύποι δεδομένων (δήλωση μεταβλητών)}
% -------------------------------------------------------

\begin{center}
\begin{tabular}{|l|l|}
\hline
\textbf{Λέξη-κλειδί} & \textbf{Περιγραφή} \\
\hline
\texttt{ΑΚΕΡΑΙΕΣ} & Δήλωση ακέραιων μεταβλητών \\
\hline
\texttt{ΠΡΑΓΜΑΤΙΚΕΣ} & Δήλωση πραγματικών μεταβλητών \\
\hline
\texttt{ΧΑΡΑΚΤΗΡΕΣ} & Δήλωση αλφαριθμητικών μεταβλητών \\
\hline
\texttt{ΛΟΓΙΚΕΣ} & Δήλωση λογικών μεταβλητών \\
\hline
\end{tabular}
\end{center}

% -------------------------------------------------------
\subsection{Τύποι δεδομένων (επιστροφή συναρτήσεων)}
% -------------------------------------------------------

\begin{center}
\begin{tabular}{|l|l|}
\hline
\textbf{Λέξη-κλειδί} & \textbf{Περιγραφή} \\
\hline
\texttt{ΑΚΕΡΑΙΑ} & Επιστροφή ακέραιου από συνάρτηση \\
\hline
\texttt{ΠΡΑΓΜΑΤΙΚΗ} & Επιστροφή πραγματικού από συνάρτηση \\
\hline
\texttt{ΧΑΡΑΚΤΗΡΑΣ} & Επιστροφή αλφαριθμητικού από συνάρτηση \\
\hline
\texttt{ΛΟΓΙΚΗ} & Επιστροφή λογικού από συνάρτηση \\
\hline
\end{tabular}
\end{center}

% -------------------------------------------------------
\subsection{Λογικές τιμές}
% -------------------------------------------------------

\begin{center}
\begin{tabular}{|l|l|}
\hline
\textbf{Λέξη-κλειδί} & \textbf{Περιγραφή} \\
\hline
\texttt{ΑΛΗΘΗΣ} & Λογική τιμή «αληθής» \\
\hline
\texttt{ΨΕΥΔΗΣ} & Λογική τιμή «ψευδής» \\
\hline
\end{tabular}
\end{center}

% -------------------------------------------------------
\subsection{Αριθμητικοί τελεστές}
% -------------------------------------------------------

\begin{center}
\begin{tabular}{|l|l|l|}
\hline
\textbf{Τελεστής} & \textbf{Όνομα} & \textbf{Περιγραφή} \\
\hline
\texttt{+} & Συν & Πρόσθεση \\
\hline
\texttt{-} & Πλην & Αφαίρεση (ή αρνητικό πρόσημο) \\
\hline
\texttt{*} & Επί & Πολλαπλασιασμός \\
\hline
\texttt{/} & Διά & Διαίρεση (αποτέλεσμα πραγματικός) \\
\hline
\texttt{\^{}} & Δύναμη & Ύψωση σε δύναμη \\
\hline
\texttt{DIV} & Ακέραια διαίρεση & Πηλίκο ακέραιας διαίρεσης \\
\hline
\texttt{MOD} & Υπόλοιπο & Υπόλοιπο ακέραιας διαίρεσης \\
\hline
\end{tabular}
\end{center}

% -------------------------------------------------------
\subsection{Συγκριτικοί τελεστές}
% -------------------------------------------------------

\begin{center}
\begin{tabular}{|l|l|l|}
\hline
\textbf{Τελεστής} & \textbf{Εναλλακτικό} & \textbf{Περιγραφή} \\
\hline
\texttt{=} & & Ίσον \\
\hline
\texttt{<>} & $\neq$ & Διάφορο \\
\hline
\texttt{<} & & Μικρότερο \\
\hline
\texttt{>} & & Μεγαλύτερο \\
\hline
\texttt{<=} & $\leq$ & Μικρότερο ή ίσο \\
\hline
\texttt{>=} & $\geq$ & Μεγαλύτερο ή ίσο \\
\hline
\end{tabular}
\end{center}

% -------------------------------------------------------
\subsection{Λογικοί τελεστές}
% -------------------------------------------------------

\begin{center}
\begin{tabular}{|l|l|l|}
\hline
\textbf{Λέξη-κλειδί} & \textbf{Αγγλικά} & \textbf{Περιγραφή} \\
\hline
\texttt{ΚΑΙ} & AND & Λογικό «και» \\
\hline
\texttt{Η} & OR & Λογικό «ή» \\
\hline
\texttt{ΟΧΙ} & NOT & Λογική άρνηση \\
\hline
\end{tabular}
\end{center}

% -------------------------------------------------------
\subsection{Εντολές εισόδου/εξόδου}
% -------------------------------------------------------

\begin{center}
\begin{tabular}{|l|l|}
\hline
\textbf{Λέξη-κλειδί} & \textbf{Περιγραφή} \\
\hline
\texttt{ΓΡΑΨΕ} & Εμφάνιση στην οθόνη \\
\hline
\texttt{ΔΙΑΒΑΣΕ} & Ανάγνωση από το πληκτρολόγιο \\
\hline
\end{tabular}
\end{center}

% -------------------------------------------------------
\subsection{Εντολή ανάθεσης}
% -------------------------------------------------------

\begin{center}
\begin{tabular}{|l|l|l|}
\hline
\textbf{Σύμβολο} & \textbf{Εναλλακτικό} & \textbf{Περιγραφή} \\
\hline
\texttt{<-} & $\leftarrow$ & Ανάθεση τιμής σε μεταβλητή \\
\hline
\end{tabular}
\end{center}

% -------------------------------------------------------
\subsection{Δομή επιλογής (ΑΝ)}
% -------------------------------------------------------

\begin{center}
\begin{tabular}{|l|l|}
\hline
\textbf{Λέξη-κλειδί} & \textbf{Περιγραφή} \\
\hline
\texttt{ΑΝ} & Έναρξη συνθήκης \\
\hline
\texttt{ΤΟΤΕ} & Μετά τη συνθήκη \\
\hline
\texttt{ΑΛΛΙΩΣ\_ΑΝ} & Εναλλακτική συνθήκη (else-if) \\
\hline
\texttt{ΑΛΛΙΩΣ} & Εναλλακτικό τμήμα (else) \\
\hline
\texttt{ΤΕΛΟΣ\_ΑΝ} & Τέλος δομής ΑΝ \\
\hline
\end{tabular}
\end{center}

% -------------------------------------------------------
\subsection{Δομή επιλογής (ΕΠΙΛΕΞΕ)}
% -------------------------------------------------------

\begin{center}
\begin{tabular}{|l|l|}
\hline
\textbf{Λέξη-κλειδί} & \textbf{Περιγραφή} \\
\hline
\texttt{ΕΠΙΛΕΞΕ} & Έναρξη πολλαπλής επιλογής \\
\hline
\texttt{ΠΕΡΙΠΤΩΣΗ} & Μία περίπτωση τιμής \\
\hline
\texttt{ΠΕΡΙΠΤΩΣΗ ΑΛΛΙΩΣ} & Προεπιλεγμένη περίπτωση \\
\hline
\texttt{ΤΕΛΟΣ\_ΕΠΙΛΟΓΩΝ} & Τέλος δομής ΕΠΙΛΕΞΕ \\
\hline
\end{tabular}
\end{center}

% -------------------------------------------------------
\subsection{Δομή επανάληψης (ΓΙΑ)}
% -------------------------------------------------------

\begin{center}
\begin{tabular}{|l|l|}
\hline
\textbf{Λέξη-κλειδί} & \textbf{Περιγραφή} \\
\hline
\texttt{ΓΙΑ} & Έναρξη επανάληψης με μετρητή \\
\hline
\texttt{ΑΠΟ} & Αρχική τιμή μετρητή \\
\hline
\texttt{ΜΕΧΡΙ} & Τελική τιμή μετρητή \\
\hline
\texttt{ΜΕ\_ΒΗΜΑ} & Βήμα μετρητή (προαιρετικό) \\
\hline
\texttt{ΜΕ ΒΗΜΑ} & Εναλλακτική γραφή (χωρίς \_) \\
\hline
\texttt{ΤΕΛΟΣ\_ΕΠΑΝΑΛΗΨΗΣ} & Τέλος βρόχου \\
\hline
\end{tabular}
\end{center}

% -------------------------------------------------------
\subsection{Δομή επανάληψης (ΟΣΟ)}
% -------------------------------------------------------

\begin{center}
\begin{tabular}{|l|l|}
\hline
\textbf{Λέξη-κλειδί} & \textbf{Περιγραφή} \\
\hline
\texttt{ΟΣΟ} & Έναρξη επανάληψης με συνθήκη \\
\hline
\texttt{ΕΠΑΝΑΛΑΒΕ} & Μετά τη συνθήκη \\
\hline
\texttt{ΤΕΛΟΣ\_ΕΠΑΝΑΛΗΨΗΣ} & Τέλος βρόχου \\
\hline
\end{tabular}
\end{center}

% -------------------------------------------------------
\subsection{Δομή επανάληψης (ΑΡΧΗ\_ΕΠΑΝΑΛΗΨΗΣ)}
% -------------------------------------------------------

\begin{center}
\begin{tabular}{|l|l|}
\hline
\textbf{Λέξη-κλειδί} & \textbf{Περιγραφή} \\
\hline
\texttt{ΑΡΧΗ\_ΕΠΑΝΑΛΗΨΗΣ} & Έναρξη επανάληψης (do-while) \\
\hline
\texttt{ΜΕΧΡΙΣ\_ΟΤΟΥ} & Συνθήκη τερματισμού \\
\hline
\end{tabular}
\end{center}

% -------------------------------------------------------
\subsection{Υποπρογράμματα}
% -------------------------------------------------------

\begin{center}
\begin{tabular}{|l|l|}
\hline
\textbf{Λέξη-κλειδί} & \textbf{Περιγραφή} \\
\hline
\texttt{ΔΙΑΔΙΚΑΣΙΑ} & Δήλωση διαδικασίας \\
\hline
\texttt{ΤΕΛΟΣ\_ΔΙΑΔΙΚΑΣΙΑΣ} & Τέλος διαδικασίας \\
\hline
\texttt{ΣΥΝΑΡΤΗΣΗ} & Δήλωση συνάρτησης \\
\hline
\texttt{ΤΕΛΟΣ\_ΣΥΝΑΡΤΗΣΗΣ} & Τέλος συνάρτησης \\
\hline
\texttt{ΚΑΛΕΣΕ} & Κλήση διαδικασίας \\
\hline
\end{tabular}
\end{center}

% -------------------------------------------------------
\subsection{Ενσωματωμένες συναρτήσεις}
% -------------------------------------------------------

\begin{center}
\begin{tabular}{|l|l|l|}
\hline
\textbf{Συνάρτηση} & \textbf{Περιγραφή} & \textbf{Τύπος αποτελέσματος} \\
\hline
\texttt{Α\_Μ(x)} & Ακέραιο μέρος & Ακέραιος \\
\hline
\texttt{Α\_Τ(x)} & Απόλυτη τιμή & Ίδιος με είσοδο \\
\hline
\texttt{Ε(x)} & Εκθετική ($e^x$) & Πραγματικός \\
\hline
\texttt{ΕΦ(x)} & Εφαπτομένη (μοίρες) & Πραγματικός \\
\hline
\texttt{ΗΜ(x)} & Ημίτονο (μοίρες) & Πραγματικός \\
\hline
\texttt{ΛΟΓ(x)} & Φυσικός λογάριθμος & Πραγματικός \\
\hline
\texttt{ΣΥΝ(x)} & Συνημίτονο (μοίρες) & Πραγματικός \\
\hline
\texttt{Τ\_Ρ(x)} & Τετραγωνική ρίζα & Πραγματικός \\
\hline
\end{tabular}
\end{center}

% -------------------------------------------------------
\subsection{Ειδικά σύμβολα}
% -------------------------------------------------------

\begin{center}
\begin{tabular}{|l|l|}
\hline
\textbf{Σύμβολο} & \textbf{Περιγραφή} \\
\hline
\texttt{!} & Σχόλιο (μέχρι τέλος γραμμής) \\
\hline
\texttt{\&} & Συνέχεια εντολής στην επόμενη γραμμή \\
\hline
\texttt{..} & Διάστημα τιμών (στην ΕΠΙΛΕΞΕ) \\
\hline
\texttt{[ ]} & Δείκτης πίνακα \\
\hline
\texttt{( )} & Παρενθέσεις (ομαδοποίηση, παράμετροι) \\
\hline
\texttt{,} & Διαχωρισμός (παραμέτρων, μεταβλητών) \\
\hline
\texttt{:} & Μετά τον τύπο δεδομένων στη δήλωση \\
\hline
\end{tabular}
\end{center}

% -------------------------------------------------------
\subsection{Συνοπτική λίστα όλων των λέξεων-κλειδιών}
% -------------------------------------------------------

Παρακάτω παρατίθενται \textbf{αλφαβητικά} όλες οι λέξεις-κλειδιά της ΓΛΩΣΣΑΣ:

\begin{center}
\begin{tabular}{|l|l|l|l|}
\hline
\texttt{ΑΛΗΘΗΣ} & \texttt{ΑΛΛΙΩΣ} & \texttt{ΑΛΛΙΩΣ\_ΑΝ} & \texttt{ΑΝ} \\
\hline
\texttt{ΑΠΟ} & \texttt{ΑΡΧΗ} & \texttt{ΑΡΧΗ\_ΕΠΑΝΑΛΗΨΗΣ} & \texttt{Α\_Μ} \\
\hline
\texttt{Α\_Τ} & \texttt{ΓΙΑ} & \texttt{ΓΡΑΨΕ} & \texttt{ΔΙΑΒΑΣΕ} \\
\hline
\texttt{ΔΙΑΔΙΚΑΣΙΑ} & \texttt{DIV} & \texttt{Ε} & \texttt{ΕΦ} \\
\hline
\texttt{ΕΠΑΝΑΛΑΒΕ} & \texttt{ΕΠΙΛΕΞΕ} & \texttt{Η} & \texttt{ΗΜ} \\
\hline
\texttt{ΚΑΛΕΣΕ} & \texttt{ΚΑΙ} & \texttt{ΛΟΓΙΚΕΣ} & \texttt{ΛΟΓΙΚΗ} \\
\hline
\texttt{ΛΟΓ} & \texttt{ΜΕ\_ΒΗΜΑ} & \texttt{ΜΕΤΑΒΛΗΤΕΣ} & \texttt{ΜΕΧΡΙ} \\
\hline
\texttt{ΜΕΧΡΙΣ\_ΟΤΟΥ} & \texttt{MOD} & \texttt{ΟΣΟ} & \texttt{ΟΧΙ} \\
\hline
\texttt{ΠΕΡΙΠΤΩΣΗ} & \texttt{ΠΡΑΓΜΑΤΙΚΕΣ} & \texttt{ΠΡΑΓΜΑΤΙΚΗ} & \texttt{ΠΡΟΓΡΑΜΜΑ} \\
\hline
\texttt{ΣΤΑΘΕΡΕΣ} & \texttt{ΣΥΝΑΡΤΗΣΗ} & \texttt{ΣΥΝ} & \texttt{ΤΕΛΟΣ\_ΑΝ} \\
\hline
\texttt{ΤΕΛΟΣ\_ΔΙΑΔΙΚΑΣΙΑΣ} & \texttt{ΤΕΛΟΣ\_ΕΠΑΝΑΛΗΨΗΣ} & \texttt{ΤΕΛΟΣ\_ΕΠΙΛΟΓΩΝ} & \texttt{ΤΕΛΟΣ\_ΠΡΟΓΡΑΜΜΑΤΟΣ} \\
\hline
\texttt{ΤΕΛΟΣ\_ΣΥΝΑΡΤΗΣΗΣ} & \texttt{ΤΟΤΕ} & \texttt{Τ\_Ρ} & \texttt{ΧΑΡΑΚΤΗΡΑΣ} \\
\hline
\texttt{ΧΑΡΑΚΤΗΡΕΣ} & \texttt{ΨΕΥΔΗΣ} & \texttt{ΑΚΕΡΑΙΕΣ} & \texttt{ΑΚΕΡΑΙΑ} \\
\hline
\end{tabular}
\end{center}

% =======================================================
% Επόμενες ενότητες θα προστεθούν εδώ
% =======================================================

\end{document}
